\documentclass{article}
\usepackage{enumitem}
\usepackage{indentfirst}
\usepackage{listings}
\usepackage{graphicx}
\usepackage{caption}
\usepackage[romanian]{babel}
\usepackage{verse}
\usepackage[a4paper,portrait,margin=1in]{geometry}
\usepackage{multicol}
\pagenumbering{gobble}

\renewcommand\thesection{\arabic{section}.}
\renewcommand\thesubsection{\thesection\arabic{subsection}.}

\newcommand{\qu}[1]{„\emph{#1}”}

\title{Alexandru Lăpușneanul}
\author{de Costache Negruzzi}
\date{}
\begin{document}
\maketitle

\part*{Comentariu}
\section{Încadrarea în context}
Nuvela \qu{Alexandru Lăpușneanul} de Costache Negruzzi a fost publicată în revista \qu{Dacia Literară} în anul 1840, apoi inclusă în volumul \qu{Păcatele tinereților} împreună cu alte nuvele. Este prima încercare de nuvelă istorică din literatura română.

Textul se încadrează în \textbf{estetica pașoptistă} prin \textbf{sursa de inspirație} reprezentată de \textbf{istoria națională}.

Pentru scrierea acestei nuvele, Negruzzi s-a inspirat din două cronici: \qu{Letopisețul Țării Moldovei} de Grigore Ureche, de unde a preluat principalele evenimente din a doua domnie a lui Lăpușneanu, și \qu{Letopisețul Țării Moldovei} de Miron Costin, de unde preia episodul linșării lui Moțoc.
\section{Titlul}
În cazul de față, titlul anticipează protagonistul nuvelei, Alexandru Lăpușneanu, domnitor al Moldovei în secolul al XVI-lea. Personajul lui Negruzzi se încadrează în \textbf{tipologia romantică a despotului} prin exagerarea unor defecte precum cruzime, autoritate, sete de putere etc.
\section{Teme}
Tema dominantă a nuvelei este una de factură romantică, și anume \textbf{evocarea unui episod din istoria națională}. Mai exact, este vorba de a doua domnie a lui Lăpușneanu cu lupta pentru putere și conflictele cu boierii.
\section{Perspectivă narativă. Narator}
Nuvela este scrisă la persoana a III-a, având o \textbf{perspectivă narativă obiectivă}. Punctul de vedere dominant îi aparține naratorului obiectiv, omniscient, omniprezent. Sunt câteva secvențe care fac excepție de la obiectivitatea naratorului, deoarece acesta devine implicat afectiv, exprimându-și punctul de vedere cu privire la situație sau la un personaj. Un exemplu este secvența \qu{Această deșănțată cuvântare} în care epitetul \qu{deșănțată} arată manifestarea subiectivă a naratorului care descrie discursul lui Lăpușneanul.
\section{Structură. Subiect}
Nuvela este structurată pe patru capitole, fiecare precedat de câte un motto care anticipează un conflict, o situație sau replica unui personaj.
\subsection{Capitolul I}
\textbf{Incipitul} nuvelei este de tip rezumativ, prezentând succint imaginea sumbră a unei Moldove sfâșiate de luptele pentru putere. În acest context tulbure, Lăpușneanul revine în țară cu sprijin turcesc pentru a-și relua tronul.

Ștefan Tomșa, actualul domnitor al Moldovei, trimite la graniță o solie formată din patru boieri pentru a-l convinge pe Lăpușneanul să se întoarcă, întrucât norodul nu-l mai vrea și nici nu-l mai iubește.

Dialogul tensionat dintre fostul domnitor și boieri anticipează un conflict mocnit care se va amplifica treptat.

\textbf{Replica memorabilă} a domnitorului \qu{Dacă voi nu mă vreți, eu vă vreu...} sugerează dorința de răzbunare împotriva boierilor care l-au trădat în prima domnie. Mimica domnitorului sugerează ura, relevantă fiind comparația \qu{ochii îi scânteiară ca un fulger}.

Moțoc este singurul boier care rămâne lângă Lăpușneanul, pentru a-l asigura de susținerea lui. Gestica și limbajul boierului îl încadrează în \textbf{tipologia lingușitorului}.

\subsection{Capitolul al II-lea}
Capitolul al II-lea se deschide cu atrocitățile făcute de domnitor imediat după urcarea sa la tron. De exemplu, Lăpușneanul arde toate cetățile Moldovei, cu excepția Hotinului pentru a împiedica eventualele comploturi ale boierilor împotriva sa; le confiscă averile pentru a le limita puterea și influența; \qu{îi omora din când în când}, cu sau fără motiv întemeiat.

Urmează o \textbf{secvență retrospectivă} din care reiese că alianța cu Ruxanda, fiica lui Petru Rareș, fost domnitor al Moldovei poate fi făcută doar din interes politic.

Reproșul care funcționează ca motto, \qu{Ai să dai samă, Doamnă!}, aparține unei jupânese al cărei soț fusese ucis de Lăpușneanul. Impresionată de lacrimile acestei jupânese, Ruxanda îl roagă pe Lăpușneanul să înceteze acest masacru sângeros. Domnitorul își controlează cu abilitate firea impulsivă și îi face Ruxandei două promisiuni: aceea că va înceta măcelul \qu{de poimâine}, și aceea că îi va găsi \qu{leac de frică}. Acest dialog pune în evidență \textbf{antiteza romantică} angelic vs. demonic, accentuând sensibilitatea Ruxandei și cinismul domnitorului.

\subsection{Capitolul al III-lea}
\textbf{Partea a III-a} și cea mai dramatică începe cu o atmosferă de sărbătoare, la mitropolie.

Domnitorul își regizează gesturile și discursul pentru a-și construi o imagine de om smerit care se căiește pentru păcatele sale. De exemplu, el se închină la toate icoanele, sărută toate moaștele și inserează citate biblice în discursul său: \qu{Bate-voi Păstorul și se vor împrăștia Oile}.

În semn de împăcare, domnitorul îi invită pe boieri la prânz, la curtea domnească. Singurii boieri sceptici, care nu cred în transformarea domnitorului sunt Spancioc și Stroici.

Momentul de maximă intensitate este măcelul celor 47 de boieri, declanșat de urarea nevinovată a lui Veveriță. În contrast cu altitudinea cinică a domnitorului care \qu{râdea în hohote} este reacția îngrozită a lui Moțoc care \qu{se sili a râde} pentru a-i face pe plac lui Lăpușneanul.

Descrierea mulțimii revoltate și debusolate care se adună la curte este \textbf{prima încercare de portret colectiv din literatura română}. Conștient de forța acestei mulțimi, Lăpușneanul îl sacrifică pe Moțoc, făcându-l responsabil pentru traiul greu al oamenilor.

Cu cinism și umor negru, Lăpușneanul îi prezintă Ruxandei piramida formată din capetele celor 47 de boieri, spunându-i că acesta este \qu{leacul de frică}.
\subsection{Capitolul al IV-lea}
\textbf{Capitolul al IV-lea} rezumă ultimii patru ani din domnia lui Lăpușneanul. Îmbolnăvindu-se, acesta se retrage la cetatea Hotin. Crezându-se pe patul de moarte, Lăpușneanul acceptă, conform obiceiului vremii, să fie călugărit pentru a i se ierta păcatele. Redobândindu-și luciditatea, îi amenință pe toți cu moartea, memorabilă fiind replica \qu{De mă voi scula, pre mulți am să popesc și eu...}.

Doamna Ruxanda este sfătuită de mitropolitul Tofan să accepte otrăvirea domnitorului, pentru a-i salva viața fiului ei.

În scena finală a otrăvirii sale, domnitorul trăiește cu intensitate umilința și revolta împotriva celor care l-au călugărit. Finalul său este unul \textbf{justificat moral}, așa cum reiese din replica boierilor: \qu{Învață a muri, tu care știai numai a omorî}.
\section{Concluzie}
În concluzie, nuvela lui Costache Negruzzi este reprezentativă pentru perioada pașoptistă, evocând a doua domnie a lui Lăpușneanu care rămâne în istorie \qu{ca o pată de sânge}.
\end{document}
