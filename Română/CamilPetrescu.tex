\documentclass{article}
\usepackage{enumitem}
\usepackage{indentfirst}
\usepackage{listings}
\usepackage{graphicx}
\usepackage{caption}
\usepackage[romanian]{babel}
\usepackage{verse}
\usepackage[a4paper,portrait,margin=1in]{geometry}
\usepackage{multicol}
\pagenumbering{gobble}

\renewcommand\thesection{\arabic{section}.}
\renewcommand\thesubsection{\thesection\arabic{subsection}.}

\newcommand{\qu}[1]{„\emph{#1}”}

\title{Ultima noapte de dragoste, întâia noapte de război}
\author{de Camil Petrescu}
\date{}
\begin{document}
\maketitle

\part*{Comentariu}
\section{Încadrarea în context}
În spațiul prozei interbelice românești, Camil Petrescu este o \textbf{prezență emblematică}, reușind o sincronizare cu tendințele prozei europene, prin preluarea și adaptarea \textbf{modelului proustian}.

Camil Petrescu consideră că \textbf{actul creației} este un act de cunoaștere, nu de invenție, care stă sub semnul \textbf{autenticității}, pe care autorul o explică în felul următor: \qu{Să nu descriu decât ceea ce văd, ceea ce aud, ceea ce înregistrează simțurile mele, ceea ce gândesc eu. [...] Eu nu pot vorbi onest decât la persoana I}.

\qu{Ultima noapte de dragoste, întâia noapte de război}, publicat pentru prima dată în anul 1930, este un \textbf{roman modern, subiectiv}, prin următoarele aspecte: epicul redus, accentul mutându-se de la realitatea exterioară la cea interioară, reprezentată de conștiința personajului narator; luciditatea autoanalizei; narațiunea subiectivă; naratorul homodiegetic; conceptul de autenticitate; tehnica memoriei afective; spațiul citadin și tipul intelectualului care își problematizează existența.
\section{Titlul}
Titlul anticipează structura bipartită a cărții, sugerând cele două experiențe existențiale și cognitive ale lui Ștefan Gheorghidiu: dragostea și războiul. Simbolul nopții sugerează incertitudinea care-l devorează pe Ștefan Gheorghidiu în timpul căsniciei cu Ela. Cele două adjective, \qu{ultima} și \qu{întâia}, sugerează ciclicitatea experiențelor, faptul că individul iese dintr-o experiență și intră într-o alta.
\section{Teme}
Temele dominante ale romanului sunt sugerate de titlu. Dragostea și războiul devin pentru Ștefan \textbf{două experiențe de autocunoaștere}, deoarece personajul narator află cum se comportă și ce învață din aceste situații limită.

O altă temă este \textbf{inadaptarea intelectualului} la o lume a imposturii și a mediocrității.
\section{Perspectivă narativă. Narator}
Romanul este scris la persoana I, având o \textbf{perspectivă narativă subiectivă}. Punctul de vedere dominant aparține naratorului homodiegetic, reprezentat de Ștefan Gheorghidiu. \textbf{Perspectiva unică și subiectivă} pe de o parte conferă \textbf{autenticitate} textului, pe de altă parte relativizează adevărul întâmplărilor.
\section{Structură. Subiect}
\textbf{Subiectul romanului} este structurat pe următoarele planuri narative: \textbf{planul realității interioare}, reprezentat de conștiința personajului narator; \textbf{planul realității exterioare, obiective}, reprezentat de lumea oamenilor de afaceri, de viața mondenă, realitatea frontului. Oscilarea protagonistului între cele două planuri narative pune în evidență \textbf{tehnica modernă a contrapunctului}. Cu alte cuvinte, Ștefan analizează cu luciditate viața de cuplu, cea politică, precum și experiența războiului.

\textbf{Incipitul} este unul \textbf{abrupt}, deoarece acțiunea romanului debutează cu \textbf{prezentul frontului}, pentru ca apoi să urmeze o enormă \textbf{analepsă}, prin care este rememorată experiența iubirii.

Capitolul I pune în evidență \textbf{cele două planuri temporale} ale discursului narativ: \textbf{timpul narării}, asociat cu prezentul frontului și \textbf{timpul narat}, asociat cu trecutul poveștii de iubire.

\textbf{La popota ofițerilor}, Ștefan Gheorghidiu asistă la o discuție contradictorie despre dragoste și fidelitate în cuplu, pornind de la un fapt divers relatat într-un articol: un bărbat care și-a ucis soția infidelă este achitat de pedeapsă. În acest context, Ștefan formulează \textbf{un punct de vedere radical}, conform căruia \qu{acei care se iubesc au drept de viață și de moarte unul asupra celuilalt}. Subiectul acestui articol reprezentativ reprezintă \textbf{stimulul memoriei afective} a protagonistului, trezindu-i amintirile legate de cei doi ani și jumătate de căsnicie cu Ela.

\textbf{Intriga romanului} pune universul suflelesc al protagonistului sub zodia incertitudinii, așa cum el însuși mărturisește: \qu{Eram însurat de doi ani și jumătate cu o colegă de la Universitate și bănuiam că mă înșală}. Gheorghidiu, pe atunci student la Filozofie, se căsătorește cu Ela, studentă la Litere. După căsătorie, cei doi tineri duc un trai modest, dar sunt fericiți.

Echilibrul cuplului este tulburat de \textbf{moștenirea} pe care Ștefan o primește la moartea unchiului Tache. Ela se implică în discuțiile despre bani, ceea ce lui Ștefan îi displace profund, deoarece o voia mereu feminină, \qu{deasupra acestor discuții vulgare}. Ela este atrasă de noul stil de viață al cuplului, unul monden, distanțându-se de idealul feminin al lui Ștefan.

La una dintre petrecerile organizate de Anișoara, verișoara lui Ștefan, Ela îl cunoaște pe \textbf{Grigoriade}, un tânăr șarmant cu studii de jurnalism și un foarte talentat dansator.

În timpul unei \textbf{excursii la Odobești}, lui Ștefan i se pare că Ela îi acorđă o atenție deosebită lui Grigoriade. Protagonistul observă gesturile și reacțiile Elei în preajma lui G., apoi le analizează și le interpretează, considerând că fac parte din intimitatea unui cuplu. De exemplu, Ela gustă din felul de mâncare al lui G., solicită amândoi aceeași melodie, Ela își schimbă starea când G. acordă atenție unei alte femei.

\textbf{O secvență relevantă pentru zbuciumul lăuntric} al protagonistului și nevoia lui de certitudini este aceea în care, întors pe neașteptate de pe front, Ștefan nu o găsește acasă pe Ela. Aceasta se întoarce dimineața, fără să ofere explicații atunci când este chestionată. Ulterior, Ștefan găsește printre cărți un bilet de la Anișoara prin care o ruga pe Ela să vină să stea cu ea pentru că Iorgu, soțul Anișoarei, era plecat. Căutător al adevărului, Ștefan probează cele susținute în bilet, discutând cu Iorgu și cu Anișoara.

Aflat într-o scurtă permisie la Câmpulung, pe Ștefan îl deranjează interesul Elei pentru \textbf{situația ei testamentară}. Gheorghidiu devine din nou torturat de incertitudini când îl zărește pe G. în oraș. Se liniștește aparent când Ela îi arată corespondența, printre scrisori fiind și cea de la Grigoriade. Discuția pe care o are cu superiorul său îl tulbură din nou pe protagonist. Colonelul este convins că G. a venit în oraș pentru o femeie, mai ales că i-a mărturisit că nu avea la cine să stea.

\textbf{Drama colectivă a războiului} lasă în urmă drama individuală a iubirii. Confruntarea directă cu moartea, scenele șocante îl obligă pe Ștefan să se raporteze la soldații din subordine, depășindu-și egoismul. Războiul este \textbf{demitizat}, fiind prezentat din perspectiva participantului pe front. Astfel, războiul este monstruos și absurd pentru că înseamnă frică de moarte, foame, frig, haos, dotare militară precară, pregătire superficială.

Întors la București, Ștefan citește cu indiferență biletul anonim în care i se dezvăluie trădarea Elei. Obosit să mai caute certitudini, Ștefan decide să se despartă de Ela, lăsându-i ei amintirile, \qu{tot trecutul}.
\section{Concluzie}
În concluzie, valorificând concepte teoretice la modă, romanul \qu{Ultima noapte de dragoste, întâia noapte de război} prezintă traseul devenirii protagonistului de la bărbatul care crede în idealul de iubire, la bărbatul indiferent, dezamăgit.

\part*{Caracterizare}
\setcounter{section}{0}
\section{Introducere}
Prin romanele sale, Camil Petrescu impune în proza românească interbelică \textbf{o nouă tipologie umană}. Personajele sale trăiesc în lumea ideilor pure, sunt \textbf{intelectuali hipersensibili, însetați de absolut}, de adevăr, trăind sub zodia lucidității deoarece \qu{câtă luciditate ... atâta dramă}.

Ștefan Gheorghidiu, protagonistul romanului \qu{Ultima noapte de dragoste, întâia noapte de război}, reprezintă \textbf{tipul intelectualului care își problematizează existența}. Acest personaj este \textbf{un inadaptat al epocii sale} pentru că nu acceptă compromisul.
\section{Cuprinsul}
În caracterizarea lui Ștefan Gheorghidiu, autorul folosește \textbf{tehnici specifice prozei analitice}: introspecția, monologul interior, memoria afectivă.

Protagonistul este marcat de \textbf{două experiențe de cunoaștere} a lumii care devin și experiențe de autocunoaștere, de îmbogățire a sinelui: dragostea și războiul. Prima este trăită sub semnul incertitudinii și reprezintă un zbucium permanent.

Ștefan face parte din categoria \textbf{inadaptaților superiori}, conștienți de superioritatea lor intelectuală și morală, în raport cu o lume mediocră, incultă și pragmatică. Ștefan aplică tiparul său de idealitate realității, partenerei de cuplu, oamenior din jur. Aceștia nu corespund exigențelor sale, de aici rezultă \textbf{drame ale incompatibilității}.

Trăsătura distinctivă a lui Ștefan este \textbf{orgoliul} care se manifestă atât în viața publică, cât și în cea privată. De exemplu, la popota ofițerilor, Ștefan manifestă o \textbf{atitudine superioară}, \qu{privindu-i disprețuitor} pe ceilalți și spunându-le \qu{Discutați mai bine ceea ce vă pricepeți}. Ștefan este inflexibil, exprimând un punct de vedere \textbf{radical}, conform căruia \qu{acei care se iubesc au drept de viață și de moarte, unul asupra celuilalt}.

Pentru Ștefan, iubirea este o experiență de autocunoaștere. Protagonistul trăiește iluzia iubirii, nu sentimentul ca atare căruia nu i se abandonează niciodată. Pentru Ștefan, o iubire mare este un proces de autosugestie, așa cum explică el la popota ofițerilor.

Polemica de la popotă declanșează memoria afectivă a protagonistului, trezindu-i amintiri legate de cei doi ani și jumătate de căsnicie cu Ela. La o autoanaliză lucidă, Ștefan recunoaște că la baza sentimentului de dragoste față de Ela a stat \textbf{orgoliul} de a fi văzut la brațul celei mai frumoase fete din Universitate.

Ștefan are \textbf{orgoliul unui Pygmalion} care o creează pe Galateea după propriul său tipar de perfecțiune, după cum el însuși mărturisește: \qu{Fată dragă, destinul este și va fi schimbat prin mine}. Împrejurările moștenirii i-o dezvăluie pentru prima dată pe Ela într-o altă lumină. Ștefan asistă la transformarea celei preaiubite într-o femeie interesată de aspectul material al existenței, îndepărtându-se de idealul său feminin: \qu{Vedeam cum femeia mea se înstrăina zi de zi, în toate preocupările și admirațiile ei de mine}.

\textbf{Hipersensibil}, Ștefan suferă deoarece raportează realitatea la ideal, la absolut. În plus, o altă sursă de suferință este \textbf{firea analitică} care îl face să analizeze, în planul conștiinței, orice gest, reacție a celor din jur. \textbf{Ela} îi reproșează lucrul acesta, \textbf{caracterizându-l direct} în felul următor: \qu{Ești de o sensibilitate imposibilă}.

\textbf{Ștefan este caracterizat indirect prin relație cu Ela}. Eroina nu poate fi percepută în mod obiectiv de cititor, pentru  că trăsăturile ei de personalitate sunt fixate exclusiv din perspectiva lui Ștefan. Acesta îi atribuie Elei trăsături contradictorii: \textbf{nu Ela se schimbă, explică Nicolae Manolescu, ci felul în care o vede Ștefan}. Tot comportamentul ei este mediat de bănuielile naratorului-personaj. De aceea, cititorul nu se poate pronunța asupra infidelității ei.

\textbf{Fire analitică}, Ștefan observă, analizează și interpretează gesturile și reacțiile Elei în preajma lui Grigoriade, \textbf{la Odobești}. Din perspectiva protagonistului, gesturile par a fi desprinse din intimitatea unui cuplu: Ela gustă din felul de mâncare al lui G., comandă amândoi același desert, solicită amândoi aceeași melodie, Ela își schimbă starea când G. acordă atenție altei femei.

Eroul lui Camil Petrescu este născut din \textbf{îndoieli, scepticism și tensiune intelectuală} întrucât el caută veșnic dovezi, certitudini. De exemplu, Ștefan probează autenticitatea biletului găsit printre cărți discutând cu Iorgu și cu Anișoara.

\textbf{Războiul} a reprezentat pentru protagonist o altă situație limită care i-a permis să se cunoască mai bine. Drama colectivă a războiului lasă în urmă drama individuală a iubirii. Confruntarea directă cu moartea, scenele șocante îl obligă pe slt. Ștefan Gheorghidiu să se raporteze la soldații din subordine, depășindu-și în felul acesta egoismul.

\textbf{Obosit} să mai caute certitudini, Ștefan decide să se despartă de Ela. Vindecat de trecut, acum o poate privi pe Ela \qu{cu indiferența cu care privești un tablou}.

\textbf{Drama eroului} se consumă pe fundalul unei societăți mediocre, dominate de instinctul de proprietate și de dorința de parvenire. Unchiul Nae, de exemplu, manifestă un \textbf{dispreț profund față de cultură, caracterizându-l direct} pe Ștefan în felul următor: \qu{Cu filozofia dumitale nu faci doi bani}.
\section{Concluzie}
În concluzie, Ștefan Gheorghidiu face parte din familia \qu{sufletelor tari}, create de Camil Petrescu. Rămâne un personaj exemplar pentru \textbf{categoria inadaptaților care refuză abdicarea de la ideal}.
\end{document}
