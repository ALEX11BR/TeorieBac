\documentclass{article}
\usepackage{enumitem}
\usepackage{indentfirst}
\usepackage{listings}
\usepackage{graphicx}
\usepackage{caption}
\usepackage[romanian]{babel}
\usepackage{verse}
\usepackage[a4paper,portrait,margin=1in]{geometry}
\usepackage{multicol}
\pagenumbering{gobble}

\renewcommand\thesection{\arabic{section}.}
\renewcommand\thesubsection{\thesection\arabic{subsection}.}

\newcommand{\qu}[1]{„\emph{#1}”}

\title{În grădina Ghetsemani}
\author{de Vasile Voiculescu}
\date{}
\begin{document}
\maketitle

\settowidth{\versewidth}{Păreau că vor să fugă din loc, să nu-l mai vadă...}
\begin{verse}[\versewidth]
Iisus lupta cu soarta și nu primea paharul... \\
Căzut pe brânci în iarbă, se-mpotrivea întruna. \\
Curgeau sudori de sânge pe chipu-i alb ca varul \\
Și-amarnica-i strigare stârnea în slăvi furtuna. \\!

O mână nendurată, ținând grozava cupă, \\
Se coboară-miindu-l și i-o ducea la gură... \\
Și-o sete uriașă stă sufletul să-i rupă... \\
Dar nu voia s-atingă infama băutură. \\!

În apa ei verzuie jucau sterlici de miere \\
Și sub veninul groaznic simțea că e dulceață... \\
Dar fălcile-ncleștându-și, cu ultima putere \\
Bătându-se cu moartea, uitase de viață! \\!

Deasupra fără tihnă, se frământau măslinii, \\
Păreau că vor să fugă din loc, să nu-l mai vadă... \\
Treceau bătăi de aripi prin vraiștea grădinii \\
Și uliii de seară dau roate după pradă. \\!
\end{verse}

\section{Încadrarea în context}
Poezia \qu{În grădina Ghetsemani} de Vasile Voiculescu face parte din volumul \qu{Pârgă}, publicat pentru prima dată în anul 1921.

Poezia se încadrează în \textbf{direcția tradiționalistă} prin tema de factură religioasă. Scriitorii tradiționa\-liști consideră că literatura trebuie să reflecte specificul etnic al unui popor, specific etnic dat de \textbf{istoria națională}, \textbf{folclor} și \textbf{componenta spirituală} a sufletului țărănesc. 

\section{Titlul}
În cazul de față, titlul anticipează un topos su semnificații simbolice. \textbf{În sens denotativ}, grădina este cadrul fizic al rugăciunii lui Iisus înainte ca acesta să fie arestat de soldații Imperiului Roman. \textbf{În sens conotativ}, grădina devine un spațiu al purificării lui Iisus de patimi, prin virtuți. 

\section{Teme}
Tema dominantă a poeziei este \textbf{ruga lui Iisus} în grădina Ghetsemani, de pe Muntele Măslinilor. În poezia de factură religioasă a lui Voiculescu, imaginea lui Iisus este asociată cu \textbf{tema patimilor}, a suferințelor îndurate.

\section{Secvențe poetice}
Poezia este structurată pe patru strofe, primele trei descriind \textbf{planul realității interioare}, al trăirilor lui Iisus, iar ultima strofă prezintă \textbf{planul realității exterioare}, al cadrului natural.

\subsection{Strofa I}
Strofa I vorbește despre \textbf{natura duală} a lui Iisus, compusă din latura umană și cea divină. Strofa debutează cu ezitarea acestuia în fața destinului de crucificat, sugestive fiind verbele \qu{lupta}, \qu{nu primea}, \qu{se-mpotrivea}. \qu{Paharul} este metafora care desemnează destinul prestabilit al lui Iisus, cel de crucificat.

Natura duală a lui Iisus este redată și printr-un contrast cromatic: epitetul \qu{sudori de sânge} sugerează suferințele, teama de moarte, ezitările care țin de \textbf{latura umană}; comparația \qu{chipu-i alb ca varul} traduce depășirea temerilor, trimițând la \textbf{natura divină}.

Suferințele lui Iisus cresc în intensitate, întreaga natură părând a fi străbătută de o jale metafizică, așa cum reiese din versul \qu{Și amarnica-i strigare stârnea în slăvi furtuna.}.

\subsection{Strofa a II-a}
Strofa a II-a detaliază \textbf{conflictul interior} al lui Iisus, care oscilează între respingerea și acceptarea destinului de crucificat. Metafora \qu{O mână nendurată} îl desemnează pe Dumnezeu Tatăl care își îndeamnă Fiul să accepte sacrificiul suprem. Dramatismul situației este sugerat de epitetul \qu{grozava cupă}.

\qu{Setea uriașă} care îl sfâșie pe Iisus este o metaforă care traduce necesitatea crucificării sale. Strofa a II-a se încheie cu rezistența la ispitire, sugerată de verbul la forma negativă \qu{nu voia s-atingă}.

\subsection{Strofa a III-a}
Strofa a III-a redă conflictul interior al lui Iisus sub forma \textbf{jocului dintre aparență și esență}. \textbf{Aparent}, destinul de crucificat este unul înspăimântător, idee sugerată de metaforele \qu{apa [...] verzuie} și \qu{veninul groaznic}. \textbf{În esență}, acest sacrificiu este făcut cu un scop nobil, acela al mântuirii umanității de păcate, idee sugerată de metaforele \qu{sterlici de miere} și \qu{dulceață}.

Strofa a III-a se încheie cu soluționarea conflictului interior care presupune acceptarea crucificării, așa cum reiese din versul \qu{Bătându-se cu moartea, uitase de viață!}.

\subsection{Strofa a IV-a}
Strofa a IV-a descrie cadrul natural care preia trăirile lui Iisus. Acest transfer de stări este redat stilistic cu ajutorul personificărilor \qu{se frământau măslinii}, \qu{Păreau că vor să fugă din loc}. Epitetul metaforic \qu{bătăi de aripi} sugerează prezența îngerilor care, în acest context, pot fi interpretați ca îngeri ai mântuirii, sau ca îngeri ai morții. Uliii sunt păsări de pradă care anticipează sfârșitul, moartea.

\section{Concluzie}
În concluzie, textul lui Vasile Voiculescu este reprezentativ pentru tradiționalismul interbelic prin tematica de inspirație religioasă.
\end{document}
