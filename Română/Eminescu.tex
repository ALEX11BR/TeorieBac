\documentclass{article}
\usepackage{enumitem}
\usepackage{indentfirst}
\usepackage{listings}
\usepackage{graphicx}
\usepackage{caption}
\usepackage[romanian]{babel}
\usepackage{verse}
\usepackage[a4paper,portrait,margin=1in]{geometry}
\usepackage{multicol}
\pagenumbering{gobble}

\renewcommand\thesection{\arabic{section}.}
\renewcommand\thesubsection{\thesection\arabic{subsection}.}

\newcommand{\qu}[1]{„\emph{#1}”}

\title{Luceafărul}
\author{de Mihai Eminescu}
\date{}
\begin{document}
\maketitle

\iffalse
\settowidth{\versewidth}{- Dar nici nu ştiu măcar ce-mi ceri}
\setlength{\columnsep}{0cm}
\begin{small}
\begin{multicols}{3}
\begin{verse}
A fost odată ca-n poveşti, \\
A fost ca niciodată. \\
Din rude mari împărăteşti, \\
O prea frumoasă fată. \\!

Şi era una la părinţi \\
Şi mândră-n toate cele, \\
Cum e Fecioara între sfinţi \\
Şi luna între stele. \\!

Din umbra falnicelor bolţi \\
Ea pasul şi-l îndreaptă \\
Lângă fereastră, unde-n colţ \\
Luceafărul aşteaptă. \\!

Privea în zare cum pe mări \\
Răsare şi străluce, \\
Pe mişcătoarele cărări \\
Corăbii negre duce. \\!

Îl vede azi, îl vede mâini, \\
Astfel dorinţa-i gata; \\
El iar, privind de săptămâni, \\
Îi cade draga fată. \\!

Cum ea pe coate-şi răzima \\
Visând ale ei tâmple, \\
De dorul lui şi inima \\
Şi sufletu-i se împle. \\!

Şi cât de viu s-aprinde el \\
În orişicare sară, \\
Spre umbra negrului castel \\
Când ea o să-i apară. \\!

...

Şi pas cu pas pe urma ei \\
Alunecă-n odaie, \\
Ţesând cu recile-i scântei \\
O mreajă de văpaie. \\!

Şi când în pat se-ntinde drept \\
Copila să se culce, \\
I-atinge mâinile pe piept, \\
I-nchide geana dulce; \\!

Şi din oglindă luminiş \\
Pe trupu-i se revarsă, \\
Pe ochii mari, bătând închişi \\
Pe faţa ei întoarsă. \\!

Ea îl privea cu un surâs, \\
El tremura-n oglindă, \\
Căci o urma adânc în vis \\
De suflet să se prindă. \\!

Iar ea vorbind cu el în somn, \\
Oftând din greu suspină: \\
- O, dulce-al nopţii mele domn, \\
De ce nu vii tu? Vină! \\!

Cobori în jos, luceafăr blând, \\
Alunecând pe-o rază, \\
Pătrunde-n casă şi în gând \\
Şi viaţa-mi luminează! \\!

El asculta tremurător, \\
Se aprindea mai tare \\
Şi s-arunca fulgerător, \\
Se cufunda în mare; \\!

Şi apa unde-au fost căzut \\
În cercuri se roteşte, \\
Şi din adânc necunoscut \\
Un mândru tânăr creşte. \\!

Uşor el trece ca pe prag \\
Pe marginea ferestei \\
Şi ţine-n mână un toiag \\
Încununat cu trestii. \\!

Părea un tânăr voievod \\
Cu păr de aur moale, \\
Un vânăt giulgi se-ncheie nod \\
Pe umerele goale. \\!

Iar umbra feţei străvezii \\
E albă ca de ceară - \\
Un mort frumos cu ochii vii \\
Ce scânteie-n afară. \\!

- Din sfera mea venii cu greu \\
Ca să-ţi urmez chemarea, \\
Iar cerul este tatăl meu \\
Şi mumă-mea e marea. \\!

Ca în cămara ta să vin, \\
Să te privesc de-aproape, \\
Am coborât cu-al meu senin \\
Şi m-am născut din ape. \\!

O, vin'! odorul meu nespus, \\
Şi lumea ta o lasă; \\
Eu sunt luceafărul de sus, \\
Iar tu să-mi fii mireasă. \\!

Colo-n palate de mărgean \\
Te-oi duce veacuri multe, \\
Şi toată lumea-n ocean \\
De tine o s-asculte. \\!

- O, eşti frumos, cum numa-n vis \\
Un înger se arată, \\
Dară pe calea ce-ai deschis \\
N-oi merge niciodată; \\!

Străin la vorbă şi la port, \\
Luceşti fără de viaţă, \\
Căci eu sunt vie, tu eşti mort, \\
Şi ochiul tău mă-ngheaţă. \\!

...

Trecu o zi, trecură trei \\
Şi iarăşi, noaptea, vine \\
Luceafărul deasupra ei \\
Cu razele-i senine. \\!

Ea trebui de el în somn \\
Aminte să-şi aducă \\
Şi dor de-al valurilor domn \\
De inim-o apucă: \\!

- Cobori în jos, luceafăr blând, \\
Alunecând pe-o rază, \\
Pătrunde-n casă şi în gând \\
Şi viaţa-mi luminează! \\!

Cum el din cer o auzi, \\
Se stinse cu durere, \\
Iar ceru-ncepe a roti \\
În locul unde piere; \\!

În aer rumene văpăi \\
Se-ntind pe lumea-ntreagă, \\
Şi din a chaosului văi \\
Un mândru chip se-ncheagă; \\!

Pe negre viţele-i de păr \\
Coroana-i arde pare, \\
Venea plutind în adevăr \\
Scăldat în foc de soare. \\!

Din negru giulgi se desfăşor \\
Marmoreele braţe, \\
El vine trist şi gânditor \\
Şi palid e la faţă; \\!

Dar ochii mari şi minunaţi \\
Lucesc adânc himeric, \\
Ca două patimi fără saţ \\
Şi pline de-ntuneric. \\!

- Din sfera mea venii cu greu \\
Ca să te-ascult ş-acuma, \\
Şi soarele e tatăl meu, \\
Iar noaptea-mi este muma; \\!

O, vin', odorul meu nespus, \\
Şi lumea ta o lasă; \\
Eu sunt luceafărul de sus, \\
Iar tu să-mi fii mireasă. \\!

O, vin', în părul tău bălai \\
S-anin cununi de stele, \\
Pe-a mele ceruri să răsai \\
Mai mândră decât ele. \\!

- O, eşti frumos cum numa-n vis \\
Un demon se arată, \\
Dară pe calea ce-ai deschis \\
N-oi merge niciodată! \\!

Mă dor de crudul tău amor \\
A pieptului meu coarde, \\
Şi ochii mari şi grei mă dor, \\
Privirea ta mă arde. \\!

- Dar cum ai vrea să mă cobor? \\
Au nu-nţelegi tu oare, \\
Cum că eu sunt nemuritor, \\
Şi tu eşti muritoare? \\!

- Nu caut vorbe pe ales, \\
Nici ştiu cum aş începe - \\
Deşi vorbeşti pe înţeles, \\
Eu nu te pot pricepe; \\!

Dar dacă vrei cu crezământ \\
Să te-ndrăgesc pe tine, \\
Tu te coboară pe pământ, \\
Fii muritor ca mine. \\!

- Tu-mi cei chiar nemurirea mea \\
În schimb pe-o sărutare, \\
Dar voi să ştii asemenea \\
Cât te iubesc de tare; \\!

Da, mă voi naşte din păcat, \\
Primind o altă lege; \\
Cu vecinicia sunt legat, \\
Ci voi să mă dezlege. \\!

Şi se tot duce... S-a tot dus. \\
De dragu-unei copile, \\
S-a rupt din locul lui de sus, \\
Pierind mai multe zile. \\!

...

În vremea asta Cătălin, \\
Viclean copil de casă, \\
Ce umple cupele cu vin \\
Mesenilor la masă, \\!

Un paj ce poartă pas cu pas \\
A-mpărătesii rochii, \\
Băiat din flori şi de pripas, \\
Dar îndrăzneţ cu ochii, \\!

Cu obrăjei ca doi bujori \\
De rumeni, bată-i vina, \\
Se furişează pânditor \\
Privind la Cătălina. \\!

Dar ce frumoasă se făcu \\
Şi mândră, arz-o focul; \\
Ei, Cătălin, acu-i acu \\
Ca să-ţi încerci norocul. \\!

Şi-n treacăt o cuprinse lin \\
Într-un ungher degrabă. \\
- Da' ce vrei, mări Cătălin? \\
Ia du-t' de-ţi vezi de treabă. \\!

- Ce voi? Aş vrea să nu mai stai \\
Pe gânduri totdeauna, \\
Să râzi mai bine şi să-mi dai \\
O gură, numai una. \\!

- Dar nici nu ştiu măcar ce-mi ceri, \\
Dă-mi pace, fugi departe - \\
O, de luceafărul din cer \\
M-a prins un dor de moarte. \\!

- Dacă nu ştii, ţi-aş arăta \\
Din bob în bob amorul, \\
Ci numai nu te mânia, \\
Ci stai cu binişorul. \\!

Cum vânătoru-ntinde-n crâng \\
La păsărele laţul, \\
Când ţi-oi întinde braţul stâng \\
Să mă cuprinzi cu braţul; \\!

Şi ochii tăi nemişcători \\
Sub ochii mei rămâie... \\
De te înalţ de subsuori \\
Te-nalţă din călcâie; \\!

Când faţa mea se pleacă-n jos, \\
În sus rămâi cu faţa, \\
Să ne privim nesăţios \\
Şi dulce toată viaţa; \\!

Şi ca să-ţi fie pe deplin \\
Iubirea cunoscută, \\
Când sărutându-te mă-nclin, \\
Tu iarăşi mă sărută. \\!

Ea-l asculta pe copilaş \\
Uimită şi distrasă, \\
Şi ruşinos şi drăgălaş, \\
Mai nu vrea, mai se lasă, \\!

Şi-i zice-ncet: - Încă de mic \\
Te cunoşteam pe tine, \\
Şi guraliv şi de nimic, \\
Te-ai potrivi cu mine... \\!

Dar un luceafăr, răsărit \\
Din liniştea uitării, \\
Dă orizon nemărginit \\
Singurătăţii mării; \\!

Şi tainic genele le plec, \\
Căci mi le umple plânsul \\
Când ale apei valuri trec \\
Călătorind spre dânsul; \\!

Luceşte c-un amor nespus, \\
Durerea să-mi alunge, \\
Dar se înalţă tot mai sus, \\
Ca să nu-l pot ajunge. \\!

Pătrunde trist cu raze reci \\
Din lumea ce-l desparte... \\
În veci îl voi iubi şi-n veci \\
Va rămânea departe... \\!

De-aceea zilele îmi sunt \\
Pustii ca nişte stepe, \\
Dar nopţile-s de-un farmec sfânt \\
Ce nu-l mai pot pricepe. \\!

- Tu eşti copilă, asta e... \\
Hai ş-om fugi în lume, \\
Doar ni s-or pierde urmele \\
Şi nu ne-or şti de nume, \\!

Căci amândoi vom fi cuminţi, \\
Vom fi voioşi şi teferi, \\
Vei pierde dorul de părinţi \\
Şi visul de luceferi. \\!

...

Porni luceafărul. Creşteau \\
În cer a lui aripe, \\
Şi căi de mii de ani treceau \\
În tot atâtea clipe. \\!

Un cer de stele dedesubt, \\
Deasupra-i cer de stele - \\
Părea un fulger ne'ntrerupt \\
Rătăcitor prin ele. \\!

Şi din a chaosului văi, \\
Jur împrejur de sine, \\
Vedea, ca-n ziua cea dentâi, \\
Cum izvorau lumine; \\!

Cum izvorând îl înconjor \\
Ca nişte mări, de-a-notul... \\
El zboară, gând purtat de dor, \\
Pân' piere totul, totul; \\!

Căci unde-ajunge nu-i hotar, \\
Nici ochi spre a cunoaşte, \\
Şi vremea-ncearcă în zadar \\
Din goluri a se naşte. \\!

Nu e nimic şi totuşi e \\
O sete care-l soarbe, \\
E un adânc asemene \\
Uitării celei oarbe. \\!

- De greul negrei vecinicii, \\
Părinte, mă dezleagă \\
Şi lăudat pe veci să fii \\
Pe-a lumii scară-ntreagă; \\!

O, cere-mi, Doamne, orice preţ \\
Dar dă-mi o altă soarte, \\
Căci tu izvor eşti de vieţi \\
Şi dătător de moarte; \\!

Reia-mi al nemuririi nimb \\
Şi focul din privire, \\
Şi pentru toate dă-mi în schimb \\
O oră de iubire... \\!

Din chaos, Doamne,-am apărut \\
Şi m-aş întoarce-n chaos... \\
Şi din repaos m-am născut, \\
Mi-e sete de repaos. \\!

- Hyperion, ce din genuni \\
Răsai c-o-ntreagă lume, \\
Nu cere semne şi minuni \\
Care n-au chip şi nume; \\!

Tu vrei un om să te socoţi \\
Cu ei să te asameni? \\
Dar piară oamenii cu toţi, \\
S-ar naşte iarăşi oameni. \\!

Ei numai doar durează-n vânt \\
Deşerte idealuri - \\
Când valuri află un mormânt, \\
Răsar în urmă valuri; \\!

Ei doar au stele cu noroc \\
Şi prigoniri de soarte, \\
Noi nu avem nici timp, nici loc \\
Şi nu cunoaştem moarte. \\!

Din sânul vecinicului ieri \\
Trăieşte azi ce moare, \\
Un soare de s-ar stinge-n cer \\
S-aprinde iarăşi soare; \\!

Părând pe veci a răsări, \\
Din urmă moartea-l paşte, \\
Căci toţi se nasc spre a muri \\
Şi mor spre a se naşte. \\!

Iar tu, Hyperion, rămâi \\
Oriunde ai apune... \\
Cere-mi cuvântul meu dentâi - \\
Să-ţi dau înţelepciune? \\!

Vrei să dau glas acelei guri, \\
Ca dup-a ei cântare \\
Să se ia munţii cu păduri \\
Şi insulele-n mare? \\!

Vrei poate-n faptă să arăţi \\
Dreptate şi tărie? \\
Ţi-aş da pământul în bucăţi \\
Să-l faci împărăţie. \\!

Îţi dau catarg lângă catarg, \\
Oştiri spre a străbate \\
Pământu-n lung şi marea-n larg, \\
Dar moartea nu se poate... \\!

Şi pentru cine vrei să mori? \\
Întoarce-te, te-ndreaptă \\
Spre-acel pământ rătăcitor \\
Şi vezi ce te aşteaptă. \\!

...

În locul lui menit din cer \\
Hyperion se-ntoarse \\
Şi, ca şi-n ziua cea de ieri, \\
Lumina şi-o revarsă. \\!

Căci este sara-n asfinţit \\
Şi noaptea o să-nceapă; \\
Răsare luna liniştit \\
Şi tremurând din apă \\!

Şi umple cu-ale ei scântei \\
Cărările din crânguri. \\
Sub şirul lung de mândri tei \\
Şedeau doi tineri singuri: \\!

- O, lasă-mi capul meu pe sân, \\
Iubito, să se culce \\
Sub raza ochiului senin \\
Şi negrăit de dulce; \\!

Cu farmecul luminii reci \\
Gândirile străbate-mi, \\
Revarsă linişte de veci \\
Pe noaptea mea de patimi. \\!

Şi de asupra mea rămâi \\
Durerea mea de-o curmă, \\
Căci eşti iubirea mea dentâi \\
Şi visul meu din urmă. \\!

Hyperion vedea de sus \\
Uimirea-n a lor faţă: \\
Abia un braţ pe gât i-a pus \\
Şi ea l-a prins în braţe... \\!

Miroase florile-argintii \\
Şi cad, o dulce ploaie, \\
Pe creştetele-a doi copii \\
Cu plete lungi, bălaie. \\!

Ea, îmbătată de amor, \\
Ridică ochii. Vede \\
Luceafărul. Şi-ncetişor \\
Dorinţele-i încrede: \\!

- Cobori în jos, luceafăr blând, \\
Alunecând pe-o rază, \\
Pătrunde-n codru şi în gând, \\
Norocu-mi luminează! \\!

El tremură ca alte dăţi \\
În codri şi pe dealuri, \\
Călăuzind singurătăţi \\
De mişcătoare valuri; \\!

Dar nu mai cade ca-n trecut \\
În mări din tot înaltul: \\
- Ce-ţi pasă ţie, chip de lut, \\
Dac-oi fi eu sau altul? \\!

Trăind în cercul vostru strâmt \\
Norocul vă petrece, \\
Ci eu în lumea mea mă simt \\
Nemuritor şi rece. \\!
\end{verse}
\end{multicols}
\end{small}
\fi

\section{Încadrarea în context}
\qu{Luceafărul} este considerat de Z. D. Bușulenga \qu{un poem care sintetizează gândirea artistului}.

\qu{Luceafărul} a fost publicat pentru prima dată în anul 1883, la Viena, în \qu{România Jună}.

Pentru scrierea acestui poem, Eminescu a valorificat diverse surse de inspirație: \textbf{surse folclorice} (basmele populare \qu{Fata în gradina de aur} și \qu{Miron și frumoasa fără corp}; motivul zburătorului), \textbf{surse mitologice} (Hyperion și Demiurgul), \textbf{surse filozofice} (conceptul omului de geniu preluat din filozofia schoppenhauriană).

Poemul este un text hibrid prin interferența genurilor și a speciilor literare. Textul poate fi interpretat ca o \textbf{meditație filozofică} pe tema condiției umane duale, prinse între un destin lipsit de ideal și dorința de a ieși din acest destin. În același timp, poemul poate fi interpretat ca \textbf{alegorie} pe tema condiției omului de geniu în lume. În sensul acest, Eminescu oferă cea mai elocventă explicație: \qu{Pe de o parte, geniul nu cunoaște nici moarte, numele lui scapă de noaptea uitării, pe de altă parte, aici pe pământ nu e capabil nici de a ferici pe cineva, nici de a fi fericit. El n-are moarte, dar n-are nici noroc}.

În poem se regăsesc atât elemente romantice, cât și elemente clasice. De romantism țin următoarele aspecte: temele de factură romantică, cosmogoniile, alternanța planului terestru cu cel cosmic, motivul luceafărului, al metamorfozelor. Ca influențe clasiciste recunoaștem compoziția simetrică, echilibrul compozițional, caracterul științific al ideilor filozofice.
\section{Titlul}
Titlul are un rol anticipativ, putând fi o cheie de descifrare a textului. În cazul de față, titlul anticipează condiția omului de geniu simbolizată de Luceafăr. Astrul poate reprezenta cel mai bine natura duală a geniului, care cu o față întoarsă spre pământ aspiră la iubire (\textbf{latura telurică}); cu o altă față, ascunsă simțurilor umane, năzuiește spre cunoașterea absolută (\textbf{latura cosmică}).

\section{Teme}
Temele și motivele literare sunt de factură romantică: \textbf{iubirea} și ipostazele acesteia (\textsl{\textbf{iubirea împlinită}} - între ființe compatibile, care aparțin aceluiași plan; \textsl{\textbf{iubirea neîmplinită}} - între ființe incompatibile, care aparțin unor lumi total diferite), \textbf{condiția omului de geniu} și legătura acestuia cu lumea, \textbf{cunoașterea absolută} ca ideal al omului comun.

\section{Structură}
Poemul este structurat pe 98 de strofe, împărțite pe patru tablouri, în care planul terestru alternează cu cel cosmic. Cele două planuri arată pendularea între cer și pământ a omului de geniu, prins între lumea spiritului și lumea pasiunilor (pământul).

\subsection{Tabloul I}
Tabloul I stă sub semnul basmului, plasând întâmplarea într-un timp anistoric (\qu{A fost odată ca-n povești, / A fost ca niciodată}).

Descrierea portret a fetei de împărat o unicizează, evidențiind frumusețea acesteia prin superlativul \qu{O prea frumoasă fată}. Comparația dublă \qu{Cum e Fecioara între sfinți / Și luna între stele} îi atribuie fetei noțiunea de puritate și năzuința spre absolut, spre sfere înalte.

Decorul romantic, favorabil visării, este asigurat de motivul lunii, al înserării, și al castelului.

În acest decor, fata de împărat contemplă Luceafărul de la fereastra castelului. Motivul privirii atribuie iubirii valoare cognitivă, de cunoaștere a celuilalt prin contemplare: fata \qu{Privea în zare}, \qu{El iar, privind de săptămâni}.

Motivul oglinzii sugerează spațiul virtual în care întâlnirea dintre planul cosmic și cel terestru devine posibilă.

Întâlnirea celor doi îndrăgostiți, aparținând unor lumi antagonice, se poate întâmpla doar în plan oniric, în visul fetei de împărat. Visul apare cu dublă semnificație: ca aspirație a omului comun spre ideal, și ca soluție de evadare din cotidian.

Cele două invocații ale Luceafărului au rezonanța unui descântec de chemare întru ființă, de concretizare a unui concept abstract.

Cele două întrupări ale Luceafărului, una angelică, cealaltă demonică, sugerează condiția duală a omului de geniu. Dominat de rațiune, omului de geniu îi lipsește pasiunea, capacitatea de a iubi și de a fi iubit. Prima apariție, cea angelică, este construită după canoanele frumuseții romantice: \qu{tânăr voievod / Cu păr de aur moale}. A doua apariție, cea demonică, este redată cu ajutorul epitetelor \qu{Pe negre vițele-i de păr / Coroana-i arde pare}. De fiecare dată, fata de împărat interpretează eronat paloarea feței și lucirea ochilor, considerându-le ca atribute ale morții, așa cum reiese din oximoronul \qu{Un mort frumos, cu ochii vii}. Dragostea lor este o atracție a contrariilor, așa cum reiese din antitezele \qu{eu sunt vie, tu ești mort}, \qu{eu sunt nemuritor, / Și tu ești muritoare}.

Fata de împărat refuză să-l urmeze pe Luceafăr în lumea lui, devenind conștientă de limitele condiției umane. Ea rămâne captivă în această lume mediocră, ratând șansa comuniunii cu absolutul. Cel care este capabil de sacrificiu suprem este omul de geniu, gata să schimbe veșnicia pe vremea trecătoare. Versurile care redau decizia Luceafărului de a renunța la statutul său ontologic de ființa superioară sunt \qu{Cu vecinicia sunt legat, / Ci voi să mă dezlege}.
\subsection{Tabloul II}
Tabloul II, dominat de planul terestru, detaliază idila dintre Cătălin și Cătălina, ambii reprezentând tipologia omului comun.

Portretul lui Cătălin este construit în antiteză cu cel al Luceafărului. Originea sa este una obscură (\qu{Băiat din flori și de pripas}), statutul este unul modest (\qu{Un paj ce poartă [...]/ A-mpărătesii rochii}), iar limbajul este plin de clișee verbale de proveniență rustică (\qu{arz-o focul}, \qu{acu-i acu}, \qu{să-mi dai / O gură}).

Jocul erotic inițiat de Cătălin este asociat cu o vânătoare dominată de viclenia vânătorului (\qu{Se furișează pânditor / Privind la Cătălina}).

Oscilarea fetei între planul cosmic și cel terestru sugerează dorul mistuitor de Luceafăr și tristețea conștientizării acestei iubiri imposibile: \qu{În veci îl voi iubi, și-n veci / Va rămânea departe...}.

Dacă Luceafărul îi oferise fetei de împărat un destin de excepție, Cătălin îi propune un destin comun, anonim, în mijlocul unei lumi în care toți sunt la fel, așa cum reiese din versurile \qu{Hai ș-om fugi în lume, / Doar nu s-or pierde urmele / Și nu ne-or ști de nume}.

\subsection{Tabloul III}
Tabloul III, dominat de planul cosmic, detaliază zborul Luceafărului și dialogul acestuia cu Demiurgul.

Zborul Luceafărului este un zbor înapoi în timp, în acel moment zero, dinaintea genezei. Nașterea lumii este descrisă ca o revărsare de materie, de lumină (\qu{Și din a chaosului văi, /[...] Vedea ca-n ziua cea dentâi / Cum izvorau lumine}). Increatul este descris cu ajutorul unei antiteze: \qu{Nu e nimic și totuși e}

Luceafărul recuperează starea originară, redevenind Hyperion. El îi adresează Demiurgului rugămintea de a-l dezlega de \qu{greul negrei vecinicii} în schimbul unui singur ceas de trăire adevărată. Metafora \qu{O oră de iubire} definește condiția umană care stă sub semnul efemerității, dar și al pasiunii compensatorii.

Răspunsul Demiurgului începe cu antiteza dintre destinul omului de geniu și cel al omului comun. Destinul omului comun este marcat de efemeritatea timpului, subliniată de metafora valurilor, oamenii comuni fiind conduși de noroc, clădindu-și idealuri deșarte. fiind prizonierii unui timp ciclic din care nu pot ieși.

Hyperion îl tentează pe Hyperion cu diferite ipostaze ale genialității: înțelepciunea, inspirația sau harul poetic, spiritul justițiar, ipostaza de conducător sau lider. Demiurgul lasă la urmă argumentul cel mai puternic, îndrumându-l pe Hyperion să privească spre pământ, pentru a se convinge că în lumea iubirilor efemere, pentru el nu mai există loc: \qu{Întoarce-te, te-ndreaptă / Spre-acel pământ rătăcitor / Și vezi ce te așteaptă}.

\subsection{Tabloul IV}
Tabloul IV debutează cu un pastel terestru. Este vorba despre un cadru nocturn care are un rol protector, care sintetizează aceleași motive literare din idilele eminesciene: cuplul de îndrăgostiți retrași în mijlocul naturii, codrul, lacul, teiul, luna.

Monologul adresat al lui Cătălin are tonalitatea unei confesiuni erotice, acesta exteriorizându-și sentimentele față de Cătălina: \qu{Căci ești iubirea mea dentâi / Și visul meu din urmă}. Prin urmare, Erosul capătă sens de experiență inițiatică și cognitivă, dând sens existenței efemere a omului comun.

Cea de-a treia invocație a fetei de împărat nu mai este acea chemare a unei iubiri imposibile, ci o încercare de a-și proteja fericirea fragilă, încredințându-și norocul unui astru: \qu{Pătrunde-n codru și în gând, / Norocu-mi luminează!}.

Ultima replică a Luceafărului transmite durerea îndrăgostitului a cărui iubire a fost disprețuită. Cea \qu{prea frumoasă} de odinioară este numită acum \qu{chip de lut}, epitetul sugerând condiția limitată a omului de rând. Condiția umană este caracterizată prin simbolul geometric al \qu{cercului strâmt}, fiind supusă norocului efemer, adică hazardului și devenirii.

În antiteză cu omul comun, omul de geniu se descrie ca fiind \qu{nemuritor și rece}. Dacă primul epitet se referă la atributul veșniciei, al doilea se referă la detașarea față de frământările planului terestru.
\section{Concluzie}
În concluzie, Eminescu creează un adevărat mit poetic în poemul \qu{Luceafărul}. Luceafărul este un personaj romantic, un simbol cu semnificații multiple, așa cum reiese din acest poem: stea fixă, Zburător, titan răzvrătit împotriva unei ordini prestabilite, suflet răvășit de patima iubirii, înger și demon.
\end{document}