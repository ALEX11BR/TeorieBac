\documentclass{article}
\usepackage{enumitem}
\usepackage{indentfirst}
\usepackage{listings}
\usepackage{graphicx}
\usepackage{caption}
\usepackage[romanian]{babel}
\usepackage{verse}
\usepackage[a4paper,portrait,margin=1in]{geometry}
\usepackage{multicol}
\pagenumbering{gobble}

\renewcommand\thesection{\arabic{section}.}
\renewcommand\thesubsection{\thesection\arabic{subsection}.}

\newcommand{\qu}[1]{„\emph{#1}”}

\title{Plumb}
\author{de George Bacovia}
\date{}
\begin{document}
\maketitle

\settowidth{\versewidth}{Pe flori de plumb, și-am început să-l strig---}
\begin{verse}[\versewidth]
Dormeau adânc sicriele de plumb, \\
Și flori de plumb și funerar veștmânt--- \\
Stam singur în cavou... și era vânt... \\
Și scârțâiau coroanele de plumb. \\!

Dormea întors amorul meu de plumb \\
Pe flori de plumb, și-am început să-l strig--- \\
Stam singur lângă mort... și era frig... \\
Și-i atârnau aripile de plumb. \\!
\end{verse}

\section{Încadrarea în context}
Criticul literar Lidia Bote este de părere că în lirica bacoviană \qu{oamenii, decorul, lucrurile au culoarea cenușie a plumbului fiindcă peste tot predomină tristețea și apăsarea}.

Poezia \qu{Plumb} deschide volumul omonim publicat pentru prima dată în 1916. Este o artă poetică modernă care sintetizează crezul artistic al poetului simbolist referitor la rolul artei, artistului, relația dintre artist și societate.

Poezia se încadrează în estetica simbolistă prin următoarele aspecte: teme și motive literare de factură simbolistă, cultivarea simbolului, tehnica sugestiei care pune în evidență corespondențele între realitatea exterioară și cea interioară, muzicalitatea versurilor.

\section{Titlul}
În cazul de față, titlul reprezintă laitmotivul și motivul central al poeziei. Ca laitmotiv, cuvântul conferă muzicalitate versurilor, fiind repetat de șase ori, în contexte diferite: \qu{sicrie de plumb}, \qu{flori de plumb} etc. Ca simbol plurivalent, plumbul are mai multe semnificații: din punct de vedere cromatic, cenușiul plumbului sugerează monotonia existenței; fiind un metal greu, plumbul sugerează apăsarea sufletească; folosindu-se în trecut la sigilarea sicrielor, plumbul poate conota ideea de moarte, de sfârșit.

\section{Teme}
Temele și motivele literare sunt de factură simbolistă, traducând o nouă viziune despre lume, despre artă. Întâlnim \textbf{teme} precum \textsl{\textbf{solitudinea}} (\qu{stam singur}), \textsl{\textbf{angoasa}} provocată de inadaptarea la lume și de senzația de sfârșit ce reiese din fiecare vers, \textsl{\textbf{iubirea neîmplinită}}, \textsl{\textbf{cadrul ostil}} care amplifică starea de disconfort a eului liric. Ca \textbf{motive literare}, în text funcționează \textsl{\textbf{sicriele și cavoul}} care conotează un spațiu închis; \textsl{\textbf{vântul și frigul}} care sugerează golul sufletesc.

\section{Structură. Secvențe poetice}
Poezia este structurată pe două strofe care pot fi asociate cu planurile realității: prima strofă sugerează planul realității exterioare, realitatea socială în care trăiește artistul; strofa a doua sugerează planul realității interioare, vorbind despre experiența iubirii.

Poezia are o structură simetrică dată de asemănările și diferențele dintre prima și a doua strofă (\qu{Dormeau adânc}--\qu{Dormea întors}, \qu{Și era vânt}--\qu{Și era frig}, \qu{Și scârțâiau}--\qu{Și atârnau}).

Eul liric în această poezie resimte disconfort atât în plan exterior, social, cât și în plan interior, personal. Solitudinea socială și cea erotică este văzută ca un eșec existențial.

\subsection{Strofa I}
Prima strofă conține reperele lumii exterioare. Cavoul și sicriul sunt simboluri pentru spații închise, în care eul liric se simte captiv. Dacă interpretăm poezia ca artă poetică, eul liric apare în ipostaza artistului inadaptat și izolat din cauza artei sale privită cu reticență de societate.

Strofa debutează cu verbul \qu{dormeau}, urmat de epitetul \qu{adânc}, ambele atribuite planului exterior al ființei: \qu{Dormeau adânc sicriele de plumb}, prin urmare somnul adânc, starea de letargie este atribuită societății. Repetarea epitetului metaforic \qu{de plumb} descrie o lume  împietrită, inertă, incapabilă de comunicare, de afecțiune, de înțelegere a artei simboliste.

Se conturează un câmp semantic al morții, al sfârșitului: \qu{sicrie}, \qu{funerar veștmânt}, \qu{cavou}, \qu{coroane}. Ideea de sfârșit generează starea de angoasă a eului liric.

Strofa se încheie cu o imagine auditivă ce amplifică starea de disconfort a eului liric: \qu{Și scârțâiau coroanele de plumb}.

\subsection{Strofa a II-a}
Strofa a doua descrie universul interior al eului liric, referindu-se la neîmplinirea în dragoste. Strofa debutează tot cu verbul \qu{dormea}, de data aceasta atribuit amorului. Epitetul \qu{întors} trimite la întoarcerea cu fața spre apus care, în mitologia populară, înseamnă trecerea la lumea de dincolo. De aici deducem ideea de sfârșit, de final al iubirii. Epitetul metaforic \qu{amorul meu de plumb} sugerează neîmplinirea în dragoste. Verbele \qu{Și am început să-l strig} traduc încercarea disperată a eului liric de a resuscita amorul. \qu{Aripile de plumb} ale amorului amintesc de zborul în jos al lui Icar ca metaforă pentru prăbușirea în singurătate a ființei umane.

\section{Concluzie}
În concluzie, \qu{Plumb} de George Bacovia este reprezentativ pentru simbolismul românesc prin tematică, prin tehnici artistice, viziunea despre lume. Poezia vorbește despre un eu captiv, pentru care soluția de salvare se amână.
\end{document}
