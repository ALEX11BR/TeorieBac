\documentclass{article}
\usepackage{enumitem}
\usepackage{indentfirst}
\usepackage{listings}
\usepackage{graphicx}
\usepackage{caption}
\usepackage[romanian]{babel}
\usepackage{verse}
\usepackage[a4paper,portrait,margin=1in]{geometry}
\usepackage{multicol}
\pagenumbering{gobble}

\renewcommand\thesection{\arabic{section}.}
\renewcommand\thesubsection{\thesection\arabic{subsection}.}

\newcommand{\qu}[1]{„\emph{#1}”}

\title{Flori de mucigai}
\author{de Tudor Arghezi}
\date{}
\begin{document}
\maketitle

\settowidth{\versewidth}{Şi m-am silit să scriu cu unghiile de la mâna stângă.}
\begin{verse}[\versewidth]
Le-am scris cu unghia pe tencuială \\
Pe un părete de firidă goală, \\
Pe întuneric, în singurătate, \\
Cu puterile neajutate \\
Nici de taurul, nici de leul, nici de vulturul \\
Care au lucrat împrejurul \\
Lui Luca, lui Marcu şi lui Ioan. \\
Sunt stihuri fără an, \\
Stihuri de groapă, \\
De sete de apă \\
Şi de foame de scrum, \\
Stihurile de acum. \\
Când mi s-a tocit unghia îngerească \\
Am lăsat-o să crească \\
Şi nu mi-a crescut -- \\
Sau nu o mai am cunoscut. \\!

Era întuneric. Ploaia bătea departe, afară. \\
Şi mă durea mâna ca o ghiară \\
Neputincioasă să se strângă \\
Şi m-am silit să scriu cu unghiile de la mâna stângă. \\!
\end{verse}

\section{Încadrarea în context}
Tudor Arghezi este cunoscut în critica literară drept inovatorul limbajului poetic, contribuind la sincronizarea literaturii române cu cea occidentală. Criticul literar Eugen Lovinescu afirmă despre Arghezi că \qu{începe o nouă estetică, estetica poeziei scoase din mizeria reziduurilor umane}.

Poezia \qu{Flori de mucigai} face parte din volumul omonim, publicat pentru prima dată în 1931.

Textul poate fi citit ca o artă poetică modernistă, deoarece sintetizează crezul artistic al autorului, vorbind despre condiția artistului, despre legătura acestuia cu lumea, despre o nouă tehnică de creație: estetica urâtului. În poezie se întâlnesc particularitățile modernismului arghezian: inovarea surselor de inspirație, limbajul șocant, dezarticularea sintaxei, cultivarea oximoronului, înnoirile prozodice.

\section{Titlul}
În cazul de față, titlul anticipează tehnica de creație folosită de autor, și anume estetica urâtului. Această tehnică presupune transfigurarea artistică a urâtului cotidian, existențial. În poezia de față, experiența carcerală devine sursă de inspirație. Din punct de vedere stilistic, titlul constituie un oximoron rezultat prin alăturarea a doi termeni incompatibili: \qu{florile} simbolizează binele, frumosul din lume, iar \qu{mucigaiul} simbolizează partea întunecată, neplăcută, urâtă a existenței. Prin alăturarea celor două simboluri, autorul intenționează să transmită un mesaj salvator: în ciuda degradării morale, sociale, umanitatea se poate salva, mai are o șansă.

\section{Teme}
Fiind o artă poetică modernă, temele sunt specifice: condiția artistului neînțeles, raportul acestuia cu societatea, noua tehnică de creație, sursele inedite de inspirație poetică.

\section{Structură. Secvențe poetice}
Înnoirile prozodice sunt vizibile și în textul acesta: secvențe poetice inegale ca întindere, măsura inegală a versurilor, îmbinarea mai multor tipuri de rimă într-o secvență poetică.

Poezia este structurată pe două secvențe poetice aparținând unor planuri distincte: \textbf{secvența I} aparține planului realității interioare, iar \textbf{secvența II} planului realității exterioare, sociale.

În text predomină un lirism subiectiv, care conferă poeziei un aer de confesiune.

\subsection{Secvența I}
Secvența I vorbește despre experiența carcerală și despre crezul artistic. Eul liric apare în ipostaza de artist silit să se manifeste artistic într-o situație limită, cum este închisoarea. De aceea, este prezent un câmp semantic al claustrării: \qu{firidă goală}, \qu{întuneric}, \qu{singurătate}.

În captivitate, artistul este lipsit de grația divină a inspirației, idee sugerată de epitetul \qu{cu puterile neajutate}.

Referința biblică este sugestivă pentru ideea manifestării artistice imposibile într-un spațiu închis. Asupra apostolilor (Luca, Marcu, Ioan) a coborât harul divin, prefigurat în simboluri animaliere (taurul, leul, vulturul). Eul liric se disociază de apostoli prin absența iluminării.

Versul \qu{Sunt stihuri fără an} sugerează ideea că arta este eternă, atemporală.

Epitetele metaforice \qu{de groapă}, \qu{de sete de apă}, \qu{de foame de scrum} trimit la sursele de inspirație ale poeziei moderne care valorifică artistic urâtul existențial. Versul \qu{Stihurile de acum} sugerează noua rețetă de creație valorificată de poezia contemporană, și anume estetica urâtului. Trimiterea la sfera senzorialului (foame, sete) conturează imaginea omului ca ființă biologică, limitată, imperfectă.

O altă trăsătură a modernismului arghezian se referă la limbajul șocant, constând în asocierea unor termeni din registre diferite, cum avem în cazul metaforei \qu{unghia îngerească}. Verbul \qu{s-a tocit} accentuează din nou ideea imposibilității manifestării artistice a eului liric într-un spațiu închis, ostil, nefavorabil artei. Aceeași idee poetică este redată de negațiile \qu{nu a mai crescut} și \qu{nu o mai am cunoscut}.

Versul care încheie secvența I ilustrează o altă trăsătură a modernismului arghezian, și anume dezarticularea sintaxei: \qu{Sau nu o mai am cunoscut}.

\subsection{Secvența II}
Secvența II conturează realitatea exterioară în termeni defavorabili. Este o lume ostilă artistului, reticentă la nou: \qu{întuneric}, \qu{ploaia bătea departe, afară}.

Verbul \qu{mă durea} și comparația \qu{mâna ca o ghiară} traduc eforturile și obstacolele artistului în încercarea de a face cunoscută o nouă metodă de creație.

Verbul \qu{m-am silit} vorbește despre atitudinea de revoltă a artistului față de societatea care nu-i înțelege inovațiile artistice. Metafora \qu{unghiile de la mâna stângă} vorbește despre opțiunea estetică finală a eului liric, opțiune adecvată pentru o existență dominată de răul și urâtul din lume.

\section{Concluzie}
În concluzie, \qu{Flori de mucigai} este un text poetic reprezentativ pentru modernismul interbelic prin limbajul șocant, inovațiile prozodice, sursele de inspirație inedite, estetica urâtului ca tehnică de creație artistică.
\end{document}
