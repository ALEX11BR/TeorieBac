\documentclass{article}
\usepackage{enumitem}
\usepackage{indentfirst}
\usepackage{listings}
\usepackage{graphicx}
\usepackage{caption}
\usepackage[romanian]{babel}
\usepackage{verse}
\usepackage[a4paper,portrait,margin=1in]{geometry}
\usepackage{multicol}
\pagenumbering{gobble}

\renewcommand\thesection{\arabic{section}.}
\renewcommand\thesubsection{\thesection\arabic{subsection}.}

\newcommand{\qu}[1]{„\emph{#1}”}

\title{Eu nu strivesc corola de minuni a lumii}
\author{de Lucian Blaga}
\date{}
\begin{document}
\maketitle

\settowidth{\versewidth}{eu cu lumina mea sporesc a lumii taină --}
\begin{verse}[\versewidth]
Eu nu strivesc corola de minuni a lumii \\
şi nu ucid \\
cu mintea tainele, ce le-ntâlnesc \\
în calea mea \\
în flori, în ochi, pe buze ori morminte. \\
Lumina altora \\
sugrumă vraja nepătrunsului ascuns \\
în adâncimi de întuneric, \\
dar eu, \\
eu cu lumina mea sporesc a lumii taină -- \\
şi-ntocmai cum cu razele ei albe luna \\
nu micşorează, ci tremurătoare \\
măreşte şi mai tare taina nopţii, \\
aşa îmbogăţesc şi eu întunecata zare \\
cu largi fiori de sfânt mister \\
şi tot ce-i neînţeles \\
se schimbă-n neînţelesuri şi mai mari \\
sub ochii mei -- \\
căci eu iubesc \\
şi flori şi ochi şi buze şi morminte. \\
\end{verse}

\section{Încadrarea în context}
Lucian Blaga este poetul-filozof, preocupat de misterele universului și de întrebările fără răspuns. Poetul Ștefan Augustin Doinaș remarca recurența termenului \qu{mister} în lirica blagiană.

Poezia \qu{Eu nu strivesc corola de minuni a lumii} face parte din volumul \qu{Poemele luminii}, publicat pentru prima dată în 1919. Metafora centrală a volumului este lumina care semnifică adevărul, revelația, cunoașterea.

Textul poate fi considerat o artă poetică modernistă, deoarece sintetizează crezul artistic al poetului cu ajutorul influențelor de factură expresionistă: tema cunoașterii, intensitatea trăirilor, subiectivitatea, lumea ca spectacol de lumini și umbre.

\section{Titlul}
Titlul are un rol anticipativ, putând fi o cheie de descifrare a textului. În cazul de față, titlul anticipează opțiunea eului liric pentru cunoașterea luciferică. Verbul la forma negativă \qu{nu strivesc} sugerează refuzul cunoașterii paradisiace, care distruge misterele, explicându-le logic. Metafora revelatorie \qu{corola de minuni a lumii} sugerează o reprezentare sferică a universului ca imagine a perfecțiunii. Una dintre caracteristicile acestei creații perfecte este prezența misterelor.

\section{Teme}
Fiind o artă poetică modernistă, temele sunt specifice: rolul artei și al artistului (fiind acela de a proteja misterele universului), raportarea la lume prin intermediul unui tipar cognitiv reprezentat de cunoașterea luciferică.

Tema de factură expresionistă este cunoașterea universului. Blaga propune două moduri de raportare a individului la lume: cunoașterea paradisiacă, minus-cunoașterea, și cunoașterea luciferică, plus-cunoașterea.

\section{Secvențe poetice}
Poezia este structurată pe trei secvențe poetice, inegale ca întindere. Măsura inegală a versurilor, versul liber și tehnica ingambamentului reprezintă alte înnoiri prozodice specifice modernismului interbelic.

Discursul liric este unul subiectiv și reflexiv, deoarece eul pledează pentru menținerea tainelor universului cu ajutorul cunoașterii luciferice.

\subsection{Secvența I}
Secvența I debutează cu reluarea titlului, accentuând în felul acesta opțiunea eului liric pentru cunoașterea luciferică.

Verbele la forma negativă \qu{nu strivesc} și \qu{nu ucid} sugerează refuzul cunoașterii paradisiace care distruge misterele lumii prin explicații și demonstrații logice, raționale.

Secvența I se încheie cu enumerația \qu{în flori, în ochi, pe buze ori morminte} ce conține metafore-simbol pentru misterele existenței umane. Florile sugerează misterul timpului efemer, ochii sugerează contemplarea ca mod de raportare la lume, buzele simbolizează cunoașterea lumii prin Logos și Eros, mormintele sugerează misterul morții ca experiență de trecere dintr-un plan al existenței într-altul.

\subsection{Secvența II}
Secvența II este structurată de o serie de antiteze: \qu{lumina altora} vs. \qu{lumina mea}, \qu{sugrumă} vs. \qu{sporesc}. Metafora \qu{lumina altora} desemnează cunoaștere paradisiacă la care apelează ceilalți oameni. Verbul \qu{sugrumă} sugerează efectul acestui tip de cunoaștere, care sugrumă tainele universului cu ajutorul gândirii raționale.

Eul liric apare în ipostaza artistului care are rolul de a proteja misterele universului cu ajutorul artei sale. Instrumentele cunoașterii luciferice la care apelează eul liric sunt metaforele, simbolurile, miturile. Acestea potențează misterele, le amplifică, le adaugă noi sensuri.

Obiectul cunoașterii este reprezentat de tainele acestei lumi. Se justifica astfel un câmp semantic al misterului din care fac parte următoarele metafore: \qu{vraja nepătrunsului ascuns}, \qu{adâncimi de întuneric}, \qu{taina nopții}.

Eul liric asociază efectul cunoașterii luciferice cu lumina selenară care nu dezvăluie decât parțial obiectele: \qu{eu cu lumina mea sporesc a lumii taină /[...] așa cum cu razele ei albe, luna mărește taina nopții}.

Prin intermediul artei sale, eul liric recreează un univers imaginar, în care detaliază propria viziune despre lume și despre rostul misterelor în această lume.

\subsection{Secvența III}
Secvența III are rol de concluzie a pledoariei eului liric pentru cunoașterea luciferică: \qu{Căci eu iubesc / și flori și ochi și buze și morminte}. Verbul \qu{iubesc} arată raportarea afectivă a eului liric la lume, motiv pentru care alege cunoașterea luciferică, singura care protejează misterele universului.

\section{Concluzie}
În concluzie, \qu{Eu nu strivesc corola de minuni a lumii} de Lucian Blaga este un text reprezentativ pentru modernismul interbelic, prin preluarea influențelor expresioniste și prin valorificarea ideilor din tratatele filozofice ale poetului.
\end{document}
