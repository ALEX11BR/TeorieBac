\documentclass{article}
\usepackage{enumitem}
\usepackage{indentfirst}
\usepackage{listings}
\usepackage{graphicx}
\usepackage{caption}
\usepackage[romanian]{babel}
\usepackage{verse}
\usepackage[a4paper,portrait,margin=1in]{geometry}
\usepackage{multicol}
\pagenumbering{gobble}

\renewcommand\thesection{\arabic{section}.}
\renewcommand\thesubsection{\thesection\arabic{subsection}.}

\newcommand{\qu}[1]{„\emph{#1}”}

\title{Povestea lui Harap-Alb}
\author{de Ion Creangă}
\date{}
\begin{document}
\maketitle

\part*{Comentariu}
\section{Încadrarea în context}
\qu{Povestea lui Harap-Alb} de Ion Creangă este un \textbf{basm cult} publicat pentru prima dată în revista \qu{Convorbiri literare} în anul 1877.

Ion Creangă valorifică în acest text \textbf{scenariul și particularitățile basmului popular}: alternarea planului real cu cel fabulos, clasificarea tipică a personajelor de basm, drumul inițiatic pe care îl parcurge protagonistul, probele, calul năzdrăvan, lupta dintre bine și rău, formulele narative etc.

\textbf{Nota de originalitate} a basmelor lui Creangă vizează: \textbf{oralitatea stilului}, \textbf{umorul} atribuit naratorului și personajelor, \textbf{umanizarea fantasticului}. \textbf{Umanizarea fantasticului} presupune faptul că cele cinci creaturi fabuloase din basmul lui Creangă vorbesc și se comportă asemenea oamenilor obișnuiți, critica literară comparându-le cu țăranii din Humulești.
\section{Titlul}
În cazul de față, titlul anticipează numele protagonistului, Harap-Alb. Sintagma \qu{Povestea lui ...} permite interpretarea basmului ca \textbf{Bildungsroman} deoarece protagonistul trece printr-un proces de formare, de maturizare.
\section{Teme}
Tema specifică basmului este \textbf{confruntarea dintre bine și rău}; cele două principii antagonice care guvernează lumea. Printre \textbf{motivele narative} specifice basmului întâlnim motivul împăratului fără urmași, superioritatea mezinului.
\section{Structură. Subiect}
Formula inițiatică cu care debutează basmul este una originală: \qu{Amu cică era odată ...}. Adverbul \qu{Amu} pune în evidență \textbf{registrul popular} al basmului, iar adverbul \qu{cică} relativizează spusele naratorului.

\textbf{Reperele spațio-temporale} sunt cele specifice basmului: diversitatea și multitudinea reperelor spațiale sugerează dificultatea aventurii protagonistului care trebuie să ajungă de la un capăt al lumii la celălalt, de la imaturitate la maturitate.

Acțiunea basmului se petrece \qu{odată}, într-un timp vag, neprecizat, ambiguu. În basme este prezent conceptul de \textbf{atemporalitate}, care constă în absența datelor istorice concrete și care are rolul de a generaliza situațiile de viață prezentate.

\textbf{Situația inițială} prezintă armonia de la curtea Craiului care avea trei feciori.

\textbf{Dezechilibrul} este provocat de scrisoarea de la Împăratul Verde, fratele Craiului, prin care îl roagă să-l trimită pe \qu{cel mai destoinic dintre feciori}, pentru a moșteni împărăția, el neavând decât fete. Este prezent aici \textbf{motivul împăratului fără urmași}.

Craiul îl va alege pe cel mai destoinic dintre feciori în urma unei \textbf{probe a curajului}. Deghizat cu o blană de urs, Craiul se ascunde sub un pod, testând curajul fiilor săi. \textbf{Simbolistica probei este una sugestivă}: \textbf{ursul} este un animal totem care simbolizează forța, curajul; \textbf{podul} simbolizează trecerea de la o etapă la alta, adică de la imaturitate la maturitate. Dezamăgit de faptul că feciorul cel mare și cel mijlociu pică proba, Craiul nu-i permite mezinului să-și încerce norocul.

Pentru mezinul Craiului urmează o \textbf{etapă de pregătire} coordonată de Sfânta Duminică, pentru a putea trece proba curajului. Deghizată în haine de cerșetoare, Sf. Duminică acceptă rolul de \textbf{mentor al mezinului}, după ce eroul a făcut dovada \textbf{milosteniei}. Sf. Duminică îl sfătuiește pe mezin cum să-și aleagă calul. Secvența este impregnată de \textbf{fabulos}: tava cu jăratic, metamorfoza calului nearătos într-un cal năzdrăvan, înzestrat cu darul vorbirii și cu puterea de a zbura \qu{ca vântul și ca gândul}.

Un alt sfat al Sf. Duminici este acela de a solicita hainele și armele din tinerețea Craiului, semn că destinul mezinului va fi asemănător cu cel al tatălui său.

Sfaturile mentorului său l-au ajutat pe mezin să treacă proba curajului. Înainte de a pleca de acasă, Craiul îi oferă și el un sfat: să se ferească de omul spân și de omul roș care, conform mentalității populare, sunt oameni răi.

O altă secvență relevantă este întâlnirea mezinului \textbf{cu Spânul}, în pădurea-labirint. \textbf{Toposul} este unul simbolic, semnificând pe de o parte moartea eroului ca mezin al Craiului și renașterea, reinventarea acestuia ca slugă a Spânului. Antagonistul profită de imaturitatea mezinului și, ieșindu-i în cale de trei ori, îi propune să-i fie ghid, cunoscând împrejurările foarte bine.

Un alt episod relevant este \textbf{coborârea în fântână}, care poate fi asociată ca simbolistică cu ritualul botezului, cele două având în comun: apa ca element regenerator, numele și noua identitate dată de Spân mezinului. Din momentul acesta, Spânul are un \textbf{dublu rol}: cel de \textbf{antagonist} și cel de \textbf{formator involuntar}, care contribuie fără să vrea la maturizarea mezinului prin probele pe care le dă. Pentru protagonistul basmului începe o \textbf{nouă etapă de formare, cea a umilinței}, în care învață, în calitate de slugă a Spânului, să devină empatic cu supușii săi. Schimbul de identități este pecetluit de un \textbf{jurământ} care probează loialitatea lui Harap-Alb.

Ajunși la curtea Împăratului Verde, Spânul îl supune pe Harap-Alb la \textbf{trei probe}: aducerea \qu{sălăților} din Grădina Ursului, aducerea pielii de cerb, bătută în nestemate și aducerea fetei Împăratului Roș. Modalitatea prin care protagonistul trece prin aceste probe ține tot de \textbf{fabulos}, prin prezența ajutoarelor și a furnizorilor care îi oferă \textbf{obiecte magice}.

A treia probă dată de Spân este mai complexă, necesitând demonstrarea altor calități. Drumul spre Împăratul Roș este marcat de \textbf{trei întâlniri semnificative}: alaiul de nuntă al furnicilor, pe care îl ajută să traverseze un pod; roiul de albine cărora le improvizează un stup; cele cinci creaturi fabuloase care pun în evidență calitățile de lider ale protagonistului. Critica literară vede în aceste creaturi fabuloase o \textbf{hiperbolizare a unor defecte omenești}. În plus, aceste creaturi fabuloase pun în evidență o particularitate a basmelor lui Creangă, și anume \textbf{umanizarea fantasticului}. Cele cinci creaturi au umor, sunt ironice, se ceartă, se împacă asemenea oamenilor obișnuiți.

La curtea Împăratului Roș, protagonistul trece printr-o altă \textbf{serie de probe} care îi \textbf{completează procesul de formare}: proba camerei de aramă este trecută cu ajutorul lui Gerilă; proba ospățului cu mâncare și băutură în exces este trecută cu ajutorul lui Setilă și Flămânzilă; la alegerea macului de nisip este ajutat de furnici; păzirea fetei și recuperarea ei este trecută cu ajutorul lui Ochilă și al lui Păsări-Lăți-Lungilă; proba dublului este trecută cu ajutorul crăiesei albinelor.

Înainte de a pleca, fata Împăratului Roș îl supune și ea la o probă pe Harap-Alb. Proba constă în aducerea unor \textbf{obiecte magice}: trei \qu{smicele de măr}, apă vie și apă moartă \qu{de unde se bat munții cap în cap}. La această probă participă animalele de companie ale personajelor, iar proba este câștigată prin vicleșug de calul năzdrăvan al lui Harap-Alb.

Când ajung la curtea Împăratului Verde, fata îl demască pe Spân, dezvăluindu-i adevărata identitate. Crezând că nu a respectat jurământul, Spânul îi retează capul lui Harap-Alb. Protagonistul trece acum prin ultima și cea mai profundă probă, cea a \textbf{coborârii în Infern}. Eroul este readus la viață de \qu{cumplita farmazoană}, cu ajutorul obiectelor magice. Cel care renaște acum este noul Harap-Alb, un împărat înțelept care deține adevărul despre viața de dincolo de moarte.

\textbf{Finalul} specific basmului este marcat de \textbf{clișeele bine cunoscute}: pedepsirea antagonistului, nunta, dobândirea statutului de împărat care marchează finalul procesului de formare.
\section{Concluzie}
În concluzie, Ion Creangă își transmite propria viziune despre lume prin intermediul subiectului și al personajelor sale. În viață, fiecare dintre noi parcurge un drum inițiatic, din care nu lipsesc probele, personajele negative și binevoitorii.
\part*{Caracterizare}
\setcounter{section}{0}
\section{Introducere}
Specificul personajelor de basm este dat de categoriile estetice întâlnite în basm, și anume \textbf{fabulosul} și \textbf{miraculosul}. De aceea, personajele de basm sunt înzestrate cu puteri supranaturale, se pot metamorfoza, sunt prezente creaturi fabuloase. 

Protagonistul basmului \qu{Povestea lui Harap-Alb} de Ion Creangă nu excelează cu nimic altceva decât cu firescul său, fiind înzestrat atât cu calități cât și cu defecte.

În literatura de specialitate, personajele de basm se clasifică în felul următor: \textbf{protagonistul} (Harap-Alb), \textbf{antagonistul} (Spânul), \textbf{ajutoarele} (Sfânta Duminică, cele cinci creaturi fabuloase), \textbf{donatorii} (Sfânta Duminică, crăiasa furnicilor, crăiasa albinelor).

Orice protagonist de basm parcurge un \textbf{traseu inițiatic} care presupune trecerea unor probe și la capătul căruia protagonistul va trece pe un plan superior al existenței, devenind împărat.
\section{Cuprinsul}
Protagonistul basmului este \textbf{caracterizat direct de narator} încă de la început, prin secvența următoare: \qu{Începe a plânge în inima sa, lovit fiind de apăsătoarele cuvinte ale tatălui său}. Din această secvență reiese sensibilitatea mezinului, rănit de refuzul Craiului, care nu i-a permis să participe la proba curajului.

\textbf{Eroul este caracterizat indirect prin relația cu Sfânta Duminică ce}, în calitate de mentor, \textbf{îi ghidează procesul de formare}. Deghizată în cerșetoare, Sfânta Duminică îi solicită, în mod repetat, un ban mezinului. Odată probată \qu{milostenia} mezinului, Sfânta Duminică acceptă să-i ghideze traseul inițiatic, oferindu-i sfaturi și obiecte magice. 

\textbf{Inocența mezinului} reiese din secvența alegerii calului. Mezinul \textbf{nu distinge aparența de esență}, tratând cu dispreț și violență calul nearătos care se apropie de tava cu jăratic.

Urmând sfaturile Sfintei Duminici, mezinul trece \textbf{proba curajului} pe care frații mai mari au ratat-o. \textbf{Simbolistica podului} este una sugestivă, semnificând trecerea mezinului de la imaturitate la maturitate.

\textbf{Naivitatea mezinului} este pusă în evidență și de întâlnirea cu Spânul. Acesta îi iese în cale de trei ori, de fiecare dată deghizat altfel, propunându-i să-i devină ghid. \textbf{Insuficienta cunoaștere a oamenilor} îl face pe mezin să cadă în capcana Spânului și să-i devină slugă. Spânul joacă un \textbf{dublu rol} prin raportare la protagonist: cel de \textsl{\textbf{antagonist}} și cel de \textsl{\textbf{formator involuntar}} care contribuie, fără să vrea, la maturizarea eroului, prin probele date. Din momentul în care \textbf{mezinul devine slugă a Spânului} începe, pentru el, un \textbf{traseu al umilinței} cu scopul de a deveni un împărat empatic cu supușii săi.

\textbf{Protagonistul este caracterizat indirect prin nume}. Din punct de vedere stilistic, numele conține un \textbf{oximoron} prin alăturarea celor două identități diferite ale eroului. Termenul \qu{Harap} sugerează originea inferioară de slugă a Spânului, în timp ce termenul \qu{Alb} sugerează originea nobilă de mezin al Craiului.

Harap-Alb este \textbf{caracterizat indirect prin faptele} pe care le săvârșește pe tot parcursul traseului inițiatic. După prima probă dată de Spân, eroul este \textbf{descurajat}. Calul îi atrage atenția că reușita nu este dată celor slabi. \textbf{Primele semne ale maturizării} încep să apară la proba a doua dată de Spân, când protagonistul învață că \textbf{suferința este dată pentru a înțelege suferința altora}.

În drum spre Împăratul Roș, Harap-Alb are trei întâlniri semnificative care contribuie la formarea sa. Astfel, protagonistul face dovada \textbf{altruismului} său, ajutând alaiul de nuntă al furnicilor să traverseze o apă. Improvizând un stup pentru un roi de albine, protagonistul demonstrează \textbf{pricepere} și \textbf{simț practic}. Întâlnirea cu cele cinci creaturi fabuloase accentuează \textbf{calitățile de lider} ale protagonistului: valorifică pe fiecare membru în parte, organizează grupul, distribuie sarcinile, mediază conflictele.

Probele date de Împăratul Roș sunt trecute cu ajutorul creaturilor, personajelor pe care le-a ajutat la rândul său. Formarea sa este incompletă fără \textbf{proba erosului}, de aceea fata Împăratului Roș îl supune și ea la o probă. Proba este câștigată prin vicleșug de calul năzdrăvan al lui Harap-Alb.

La curtea Împăratului Verde, \textbf{măștile cad}, fata dezvăluind adevărata identitate a Spânului. Bănuin\-du-l de trădare, Spânul îi retează capul lui Harap-Alb. Eroul este readus la viață de \qu{cumplita farmazoană}, cu ajutorul obiectelor magice. Cel care renaște acum este noul Harap-Alb, un \textbf{împărat înțelept} care deține adevărul despre viața de dincolo de moarte.
\section{Concluzie}
Dacă în basmele populare eroul face dovada unor puteri supranaturale, în basmul lui Creangă eroul este înzestrat cu calități și defecte. Acesta devine pe parcursul formării sale purtătorul unor valori și principii etice precum cinste, hărnicie, adevăr, bunătate.
\end{document}
