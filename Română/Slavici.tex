\documentclass{article}
\usepackage{enumitem}
\usepackage{indentfirst}
\usepackage{listings}
\usepackage{graphicx}
\usepackage{caption}
\usepackage[romanian]{babel}
\usepackage{verse}
\usepackage[a4paper,portrait,margin=1in]{geometry}
\usepackage{multicol}
\pagenumbering{gobble}

\renewcommand\thesection{\arabic{section}.}
\renewcommand\thesubsection{\thesection\arabic{subsection}.}

\newcommand{\qu}[1]{„\emph{#1}”}

\title{Moara cu noroc}
\author{de Ioan Slavici}
\date{}
\begin{document}
\maketitle

\part*{Comentariu}
\section{Încadrarea în context}
Nuvela realistă, de analiză psihologică \qu{Moara cu noroc} de Ioan Slavici este inclusă în volumul \qu{Novele din popor}, publicat pentru prima dată în 1881. Volumul este o adevărată monografie a satului transilvănean la sfârșitul secolului al XIX-lea, moment în care apar și se dezvoltă afacerile de familie.

Ioan Slavici își construiește subiectele și personajele pornind de la respectarea unor norme și principii etice ferme, precum cinste, dreptate, adevăr, unitatea familiei.

Nuvela este una de factură realistă, fiind purtătoarea unui mesaj moralizator, care îndeamnă la păstrarea echilibrului și a modestiei. Printre alte trăsături realiste care se regăsesc în nuvelă menționăm: veridicitatea întâmplărilor, tehnica detaliului, personaje tipice în împrejurări tipice, critica moravurilor sociale.
\section{Titlul}
Titlul are un rol anticipativ, putând fi o cheie de descifrare a textului. În cazul de față, titlul anticipează un topos cu semnificații simbolice. Conform mentalității populare, hanul construit pe locul unei mori părăsite, aflate la răscruce de drumuri, atrage energii nefaste, malefice. Titlul conține o \textbf{antifrază}, care contrazice așteptările cititorului, întrucât moara se dovedește a fi cu ghinion pentru membrii familiei lui Ghiță.

\section{Teme}
Fiind o nuvelă realistă, temele sunt specifice: banul, parvenirea, familia. În această nuvelă, banul are o conotație negativă, deoarece are drept consecință dezumanizarea individului.

Din perspectivă psihologică, tema vizează etapele procesului de dezumanizare a protagonistului nuvelei. Dintr-un cizmar cinstit și un soț iubitor, Ghiță se transformă într-un cârciumar avid și un soț agresiv, insensibil.

\section{Perspectivă narativă. Narator}
Ca orice operă realistă, nuvela este scrisă la persoana a III-a, punctul de vedere dominant aparținând naratorului obiectiv, omniprezent, omniscient.

Apar câteva excepții, în incipit și în final, unde întâlnim punctele de vedere ale bătrânei, soacra lui Ghiță. Intervențiile simetrice ale bătrânei sintetizează o perspectivă tradiționalistă, conservatoare despre lume.
\section{Structură. Subiect}
Nuvela este structurată pe șaptesprezece capitole, din care primul are rol de prolog, anticipând tema și conflictul dominant al nuvelei, iar ultimul are rol de epilog, accentuând mesajul moralizator al textului.

Nuvela are o \textbf{structură simetrică}, dată de relația de asemănare și diferențiere între incipit și final. Aici sunt prezente intervențiile bătrânei (soacra lui Ghiță) care accentuează \textbf{mesajul moralizator} al nuvelei.

\subsection{Incipitul}
\textbf{Incipitul} conține punctul de vedere al bătrânei cu privire la soluția fericirii: \qu{Omul să fie mulțumit cu sărăcia sa, căci nu bogăția, ci liniștea colibei tale te face fericit}. Această perspectivă asupra vieții, asupra destinului este una tradiționalistă, arhaică, și se referă la faptul că omul trebuie să fie mulțumit cu ce i se dă și că nu trebuie să-și provoace destinul.

\textbf{Răspunsul lui Ghiță} ilustrează o perspectivă dinamică asupra vieții, deschisă la schimbare. Ghiță este un cizmar cinstit, muncitor, care duce un trai modest împreună cu familia sa. Dorindu-și un trai mai bun, Ghiță decide să ia în arendă cârciuma de la Moara cu noroc. Asta se întâmplă de \textbf{Sfântul Gheorghe}, un reper religios, cu semnificații simbolice, deoarece anticipează eliminarea răului.

\subsection{Expozițiunea}
Expozițiunea este una tipic realistă, fixând acțiunea în timp și spațiu. \textbf{Spațiul} este cel transilvănean, în zona comunei Șiria, așa cum reiese din secvențele descriptive de la început: \qu{De la Ineu, drumul de țară o ia printre păduri și țarini ...}. Unele detalii din descrierea locului au rol anticipativ: \qu{drumul a fost cam greu}, \qu{locurile sunt rele}. La acestea se adaugă prezența celor cinci cruci din fața morii.

Inițial, lucrurile merg bine pentru familia lui Ghiță: clienții sunt numeroși, banii se fac repede, cârciuma devine un loc umanizat.

\subsection{Intriga}
Echilibrul familiei este perturbat de apariția lui Lică Sămădăul la cârciumă, moment ce constituie \textbf{intriga} nuvelei. Este prezentă \textbf{tehnica detaliului} în descrierea portret a lui Lică. Detaliile fizice și vestimentare au rolul de a caracteriza indirect personajul, sugerând trăsături de caracter ale acestuia. De exemplu, ochii mici și verzi sugerează răutate, sprâncenele dese și împreunate la mijloc sugerează o fire agresivă, iar atenția acordată accesoriilor vestimentare arată un individ orgolios, cu o stimă de sine sporită.

\textbf{Dialogul dintre Ghiță și Lică} pare un interogatoriu condus de Sămădău. Acesta își impune cu fermitate condițiile colaborării cu actualul cârciumar: \qu{Eu voiesc să știu totdeauna cine umblă pe drum, cine trece pe aici, cine ce zice și cine ce face}.

\subsection{Desfășurarea acțiunii}
După plecarea lui Lică, Ghiță își ia primele \textbf{măsuri de precauție}: cumpără două pistoale, doi câini pe care îi dresează să simtă de la depărtare turmele de porci și mai angajează o slugă.

\textbf{Desfășurarea acțiunii} detaliază etapele procesului de dezumanizare prin care trece cârciumarul, din cauza patimii pentru bani.

\textbf{Un prim semn} al dezumanizării cârciumarului este \textbf{transformarea} acestuia dintr-un om calm, iubitor într-un individ irascibil, agresiv cu cei din jur. Orgolios fiind, Ghiță, subordonat lui Lică, acumulează frustrări care se exteriorizează prin ieșiri violente: \qu{devenise ursuz de tot, se aprindea pentru orișice lucru de nimic}.

\textbf{Un alt semn al dezumanizării} este distanțarea față de Ana, căreia îi ascunde detaliile colaborării cu Lică. Ana observă transformările soțului, încercând să-i găsească scuze și să restabilească comunicarea. Ghiță se comportă în continuare ciudat, făcând-o pe Ana să-și piardă încrederea în el.

\textbf{Un alt factor} care contribuie la dezumanizarea cârciumarului este implicarea acestuia în afacerile ilegale ale lui Lică: jefuirea arendașului evreu, omorârea văduvei și a copilului acesteia.

\textbf{Imaginea cârciumarului} în comunitate este afectată, în momentul în care este dus la tribunal, încătușat, în calitate de martor în procesul lui Lică. Declarația evazivă a cârciumarului contribuie la eliberarea lui Lică.

\textbf{După proces}, apar primele mustrări de conștiință, Ghiță cerându-și iertare Anei și copiilor pentru că \qu{n-au avut un tată om cinstit}.

Cârciumarul nu se poate sustrage influenței malefice exercitate de Lică, care îi intuiește slăbiciunea pentru bani. În acest sens, Sămădăul îl tentează pe Ghiță cu sume de bani și porci furați. Cârciumarul oscilează între două moduri de existență contradictorii: traiul cinstit simbolizat de Ana și de valorile familiei vs. traiul necinstit simbolizat de Lică.

Lică pune în aplicare diverse \textbf{strategii} pentru a menține colaborarea cu Ghiță: amenințarea, lauda, manipularea, Sămădăul intuindu-i slăbiciunea cârciumarului pentru o singură femeie -- Ana.
Ghiță colaborează cu autoritățile, find de acord să-l predea pe Lică, nu înainte de a-și retrage partea cuvenită. Cârciumarul nu o avertizează pe Ana în privința acestui plan, aruncând-o în brațele Sămădăului.

\subsection{Punctul culminant}
Orbit de gelozie, Ghiță o ucide de Ana, secvență care constituie \textbf{punctul culminant} al nuvelei. Cârciumarul apare aici în \textbf{ipostaza de justițiar}, motivând că o salvează pe Ana de la un trai păcătos.

La rândul său, Ghiță este ucis de Răuț, care dă foc cârciumii din ordinul lui Lică. Sămădăul se sinucide, zdrobindu-și capul de un copac, din orgoliul de a nu fi prins de Pintea. Slavici își pedepsește drastic toate personajele care au făcut excese, care s-au abătut de la normele și principiile etice. În felul acesta, autorul accentuează latura moralizatoare a nuvelei.

\subsection{Deznodământul}
\textbf{Deznodământul} este marcat de ruinele fumegânde ale cârciumii, contemplate de bătrână și cei doi copii. \textbf{Focul} are o dublă semnificație: pe de o parte purifică locul de energiile malefice, pe de altă parte distruge locul transformându-l în ruine.

Replica finală a bătrânei încheie simetric nuvela: \qu{Așa le-a fost dată ...}. Așadar, replica se referă la rolul destinului prestabilit, care se dovedește a fi mai puternic decât individul.

\section{Concluzie}
În concluzie, \qu{Moara cu noroc} de Ioan Slavici este o nuvelă realistă, de analiză psihologică, în care patima oarbă pentru bani este amendată de destin.

\part*{Caracterizare}
\setcounter{section}{0}
\section{Introducere}
Ghiță este \textbf{protagonistul nuvelei} realiste de analiză psihologică \qu{Moara cu noroc} de Ioan Slavici.

Din punct de vedere al curentului literar, este un \textbf{personaj realist}, adică un personaj tipic în împrejurări tipice, care împrumută mentalitatea și comportamentul mediului din care provine. De exemplu, Ghiță preia tendința afacerilor de familie, luând în arendă cârciuma de la Moara cu noroc. Tot ca \textbf{personaj realist}, Ghiță poate fi încadrat într-o tipologie pe baza trăsăturii de caracter. Mai exact, acesta se încadrează în \textbf{tipologia parvenitului}, care decade atât spiritual, cât și moral, social.

Potrivit teoriei literare, Ghiță este un \textbf{personaj rotund}, deoarece statutul său social și moral se va schimba pe parcursul nuvelei. De exemplu, din cizmarul cinstit, muncitor, familist, Ghiță se va schimba în cârciumarul avid și înstrăinat de valorile familiei.

\section{Cuprinsul}
Personajul este caracterizat atât direct, cât și indirect prin fapte, comportament și relația cu celelalte personaje.

Ghiță este \textbf{caracterizat direct de narator} încă de la începutul nuvelei, ca \qu{om harnic și sârguitor, mereu așezat și pus pe gânduri}. Personajul își dorește un trai mai bun pentru familia sa, de aceea decide să ia în arendă cârciuma de la Moara cu noroc. Această hotărâre îl caracterizează indirect pe Ghiță ca fiind un \textbf{tip cu inițiativă, dinamic, deschis la schimbare}.

Ghiță este \textbf{caracterizat indirect prin relație cu Ana}, soția sa. Raportul dintre cei doi este unul specific unei familii tradiționale, în care autoritatea bărbatului este recunoscută de soție. Ana nu este doar o femeie frumoasă, ci și o parteneră de încredere, pentru că este înțeleaptă și pentru că știe ce sfaturi să dea. Familia este fericită și unită în tot ceea ce face.

\textbf{Echilibrul familiei este perturbat} de apariția lui Lică Sămădăul la cârciumă. Primul dialog dintre Ghiță și Lică pare un interogatoriu condus de Sămădău. Acesta își impune cu autoritate și fermitate condițiile colaborării cu noul cârciumar: \qu{Eu voiesc să știu totdeauna cine umblă pe drum, cine trece pe aici, cine ce zice și cine ce face, și nimeni în afară de mine să nu știe}.

Ana intuiește firea ascunsă a lui Lică și îl avertizează pe Ghiță să se țină departe de acesta. \textbf{Prevă\-zător}, Ghiță își ia câteva măsuri de precauție: cumpără două pistoale, doi câini pe care îi dresează să simtă de la depărtare turmele de porci și mai angajează o slugă.

Ghiță înțelege că trebuie să se pună bine cu Lică, și, \textbf{orgolios fiind}, acumulează frustrări care le exteriorizează prin \textbf{irascibilitate} și \textbf{agresivitate} față de cei din jur. \textbf{Naratorul} notează această transformare a personajului, \textbf{caracterizându-l direct} în felul următor: \qu{acum el se făcuse mai de tot ursuz, se aprindea pentru orișice lucru de nimic}.

Analizând perspectiva unei colaborări cu Lică, Ghiță regretă ca are familie și copii, un obstacol pentru Ghiță în calea îmbogățirii rapide. Din secvența \qu{Se gândea la câștigul pe care l-ar face în tovărășia lui Lică, vedea banii grămadă înaintea sa, și i se împăienjeneau parcă ochii} reiese \textbf{patima pentru bani} a protagonistului.

\textbf{O altă etapă a dezumanizării} cârciumarului este \textbf{înstrăinarea față de Ana}, căreia îi ascunde detaliile colaborării cu Lică. Ana nu renunță ușor la dragostea ei pentru Ghiță, de aceea încearcă să-i găsească scuze pentru comportamentul inadecvat, încearcă în repetate rânduri să restabilească comunicarea cu soțul ei. Ghiță rămâne în continuare distant, făcând-o pe Ana să-și piardă încrederea în el.

\textbf{Ghiță este caracterizat indirect prin faptele pe care le face în tovărășia lui Lică}. Sămădăul \textbf{îi strică} imaginea cârciumarului \textbf{în comunitate}, implicându-l în afacerile sale ilicite: jefuirea arendașului evreu și omorârea văduvei și a copilului acesteia. Sămădăul reușește să-l manipuleze pe cârciumar, întrucât îi intuiește slăbiciunea pentru bani. \textbf{Fire slabă} fiind, Ghiță nu se poate sustrage atracției malefice pe care Lică o exercită asupra sa, așa cum reiese din \textbf{monologul interior}: \qu{Ei! [...] Ce să-mi fac, dacă e în mine ceva mai tare decât voința mea?}.

Protagonistul nuvelei este marcat de un \textbf{puternic conflict interior, de natură psihologică}, care îl face să oscileze între două moduri de existență contradictorii: pe de o parte \textbf{traiul simbolizat de Ana} și de valorile familiei, pe de altă parte \textbf{traiul simbolizat de Lică}, care presupune îmbogățire rapidă, risc, abateri de la morală și de la lege.

La proces, Ghiță apare în ipostaza de \textbf{martor mincinos}, contribuind prin declarația sa ambiguă la eliberarea lui Lică. \textbf{După proces}, Ghiță are \textbf{primele mustrări de conștiință}, cerându-și iertare Anei și copiilor care \qu{n-au avut un tată om cinstit}.

De dragul familiei și de rușinea lumii, Ghiță se gândește că ar fi mai bine să plece de la Moara cu noroc. În acest sens, Ghiță începe să colaboreze cu Pintea, dar \textbf{nu este în totalitate sincer} nici cu acesta. Cârciumarul îi oferă probe în ceea ce privește vinovăția Sămădăului numai după ce își retrage partea cuvenită. De aici reiese, încă o dată, \textbf{lăcomia cârciumarului}.

Când Lică află că Ghiță vrea să renunțe la colaborare, acesta pune în aplicare diverse \textbf{strategii de control asupra celuilalt}: amenințarea, lauda (\qu{Tu ești om cu minte, Ghiță}), manipularea (slăbiciunea pentru o singură femeie).

Ghiță ajunge pe \textbf{ultima treaptă a degradării morale} atunci când o împinge pe Ana în brațele Sămădăului, sperând că va rezista tentației. Ana nu este avertizată asupra rolului pe care trebuie să-l joace, se simte abandonată de Ghiță, pe care îl consideră \textbf{laș}, așa cum rezultă din \textbf{caracterizarea directă}: \qu{Ghiță nu este decât o muiere îmbrăcată în haine bărbătești}.

Orbit de gelozie, Ghiță o ucide pe Ana, motivând că o salvează de la un trai păcătos. La rândul său, Ghiță este ucis de Răuț, omul lui Lică. Slavici își pedepsește cu moartea toate personajele care au comis excese, care s-au abătut de la normele morale, accentuând în felul acesta latura moralizatoare a nuvelei.

\section{Concluzie}
În concluzie, în cazul lui Ghiță, patima oarbă pentru bani este amendată de destin. Dar eșecul lui Ghiță nu trebuie văzut ca generat doar de patima banului, ci și de alegerile greșite pe care le face protagonistul.
\end{document}
