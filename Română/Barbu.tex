\documentclass{article}
\usepackage{enumitem}
\usepackage{indentfirst}
\usepackage{listings}
\usepackage{graphicx}
\usepackage{caption}
\usepackage[romanian]{babel}
\usepackage{verse}
\usepackage[a4paper,portrait,margin=1in]{geometry}
\usepackage{multicol}
\pagenumbering{gobble}

\renewcommand\thesection{\arabic{section}.}
\renewcommand\thesubsection{\thesection\arabic{subsection}.}

\newcommand{\qu}[1]{„\emph{#1}”}

\title{Din ceas, dedus...}
\author{de Ion Barbu}
\date{}
\begin{document}
\maketitle

\settowidth{\versewidth}{Şi cântec istoveşte: ascuns, cum numai marea}
\begin{verse}[\versewidth]
Din ceas, dedus adâncul acestei calme creste, \\
Intrată prin oglindă în mântuit azur, \\
Tăind pe înecarea cirezilor agreste, \\
În grupurile apei, un joc secund, mai pur. \\!

Nadir latent! Poetul ridică însumarea \\
De harfe resfirate ce-n zbor invers le pierzi \\
Şi cântec istoveşte: ascuns, cum numai marea \\
Meduzele când plimbă sub clopotele verzi. \\!
\end{verse}

\section{Încadrarea în context}
Poetul-matematician Dan Barbilian, cunoscut sub pseudonimul literar Ion Barbu, a revoluționat limbajul poetic din perioada interbelică, împreună cu Tudor Arghezi.

Criticul literar Tudor Vianu a afirmat despre Ion Barbu că \qu{este înzestrat cu o emotivitate puternică, cenzurată de o rațiune severă}, dată de exercițiul matematicii.

Poezia \qu{Din ceas, dedus...} face parte din volumul \qu{Joc secund}, publicat pentru prima dată în 1930.

Poezia poate fi citită ca artă poetică modernistă, deoarece sintetizează crezul artistic al poetului. În concepția lui Barbu, poezia modernă trebuie să fie ermetică, cu un limbaj codificat, cu un mesaj poetic încifrat cu ajutorul metaforelor, simbolurilor matematice, miturilor. Prin urmare, poezia se adresează unui lector avizat, solicitându-l din punct de vedere intelectual.

\section{Titlul}
Titlul are un rol anticipativ, putând fi o cheie de descifrare a textului. În cazul de față, titlul anticipează tehnica de creație folosită de Barbu, și anume ermetismul. Epitetul \qu{dedus} sugerează ideea că poezia ermetică este extrasă din realitatea cotidiană, simbolizată de \qu{ceas}.

\section{Teme}
Fiind o artă poetică modernistă, temele sunt specifice: rolul poetului, poeziei, relația poet-poezie, poet-lume.

\section{Secvențe poetice}
Cele două strofe ale poeziei au fost considerate de George Călinescu o definiție metaforică a metaforei.

\subsection{Strofa I}
Strofa I vorbește despre poezia modernă și despre rolul acesteia. Metafora \qu{adâncul acestei calme creste} sugerează faptul că poezia ermetică este un produs superior, întrucât implică gândirea, creativitatea, capacitatea de analiză și sinteză a creatorului.

Oglinda este o altă metaforă-simbol care trimite la ideea de reflecție, de trecere prin filtrul gândirii artistului a acelei realități văzute și trăite. În urma procesului de reflecție rezultă realitatea personală a artistului, transfigurată artistic în poezie.

Epitetul în inversiune \qu{mântuit azur} implică aceeași idee a produsului artistic superior, creat cu ajutorul gândirii și imaginației artistului.

Metafora \qu{cirezilor agreste} dezvăluie realitatea brută care nu rămâne în aceeași formă în poezie, ci e transfigurată artistic, e trecută prin filtrul gândurilor și al stărilor creatorului.

Oglinda apei sugerează ideea că poezia modernă nu este o copie fidelă a realității, ci o transfigurare artistică a acesteia, o realitate de gradul al doilea.

Epitetul dublu \qu{joc secund, mai pur} este o definiție sintetică a poeziei ermetice. Jocul presupune asocierea inedită a cuvintelor și a ideilor. Epitetul \qu{secund} trimite la ideea de realitate prelucrată, trecută prin filtrul rațional al artistului. Epitetul \qu{mai pur} semnifică utilizarea unui limbaj purificat, comparativ cu limbajul comun, cotidian.

\subsection{Strofa II}
Strofa II vorbește despre rolul poetului și viziunea sa despre poezia modernă.

Metafora \qu{Nadir latent!} conturează imaginarul artistic în formă latentă, gata oricând să se exteriorizeze în scris.

Rolul poetului este acela de a sintetiza prin arta sa aspectele esențiale ale existenței, așa cum reiese din secvența \qu{Poetul ridică însumarea / De harfe răsfirate}.

\qu{Zborul invers} este o metaforă care vorbește despre etapele procesului de creație, ce trebuie parcurse într-o ordine: mai întâi poetul trăiește experiența, și apoi o transformă în sursă de inspirație, nu invers.

Verbul \qu{istovește} trimite la efortul creator care-l consumă pe artist.

Strofa a doua se încheie cu o comparație care încifrează mesajul poetic: \qu{cum numai marea / Meduzele când plimbă sub clopotele verzi}. Comparația se referă la faptul că orice produs artistic transformă realitatea, așa cum claritatea apei mării este modificată de prezența meduzelor.

\section{Concluzie}
În concluzie, \qu{Din ceas, dedus...} de Ion Barbu este o poezie reprezentativă pentru modernismul interbelic, deoarece vorbește despre poet și poezie cu ajutorul unei noi tehnici de creație, aceea reprezentată de ermetism.
\end{document}
