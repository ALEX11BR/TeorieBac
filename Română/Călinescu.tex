\documentclass{article}
\usepackage{enumitem}
\usepackage{indentfirst}
\usepackage{listings}
\usepackage{graphicx}
\usepackage{caption}
\usepackage[romanian]{babel}
\usepackage{verse}
\usepackage[a4paper,portrait,margin=1in]{geometry}
\usepackage{multicol}
\pagenumbering{gobble}

\renewcommand\thesection{\arabic{section}.}
\renewcommand\thesubsection{\thesection\arabic{subsection}.}

\newcommand{\qu}[1]{„\emph{#1}”}

\title{Enigma Otiliei}
\author{de George Călinescu}
\date{}
\begin{document}
\maketitle

\part*{Comentariu}
\section{Încadrarea în context}
George Călinescu continuă tradiția \textbf{romanului obiectiv, de inspirație socială}. În cazul romanului \qu{Enigma Otiliei}, publicat în anul 1938, autorul folosește \textbf{formula romanului realist, de factură balzaciană}, care radiografiază burghezia și caracterele umane. Romanul este unul \textbf{realist} prin următoarele \textbf{aspecte}: veridicitatea întâmplărilor, tematica de factură realistă, descrierile detaliate care au rolul de a caracteriza indirect personajele, critica moravurilor sociale, obiectivitatea și finalul închis.

Criticul literar Nicolae Manolescu este de părere că autorul abordează balzacianismul într-un mod polemic. În primul rând, obiectivitatea romanului este încălcată prin intervențiile unor personaje, din perspectiva cărora se prezintă întâmplările. În al doilea rând, romanul se distanțează de formula realistă, de factură balzaciană, prin interesul autorului pentru psihologiile deviante (Simion Tulea -- senilul, Titi Tulea -- are un retard intelectual și afectiv).
\section{Titlul}
Titlul inițial al romanului a fost \qu{Părinții Otiliei}, titlu care punea în evidență tema realistă a paternității. La sugestia editorului său, George Călinescu schimbă titlul romanului în \qu{Enigma Otiliei}. A doua variantă de titlu sugerează \textbf{un aspect al modernismului} romanului: tehnica de caracterizare folosită în cazul Otiliei, \textbf{reflectarea poliedrică}.
\section{Teme}
\textbf{Temele} romanului sunt \textbf{de factură realistă}: observarea burgheziei bucureștene de la începutul secolului al XX-lea, moștenirea, familia, paternitatea, banul. Banul are aici o conotație negativă, deoarece pune în evidență o societate dezumanizată, degradată moral.

O altă temă este \textbf{iubirea}, surprinsă în diverse ipostaze: \textsl{\textbf{iubirea adolescentină}} în cazul cuplului Felix--Otilia; \textsl{\textbf{iubirea matură}} a lui Pascalopol pentru Otilia, \textsl{\textbf{absența iubirii}} din anticuplul Olimpia--Stănică Rațiu.

\section{Perspectivă narativă. Narator}
Fiind un roman realist, este scris la persoana a III-a, având o \textbf{perspectivă narativă obiectivă}. Deși punctul de vedere dominant aparține naratorului impersonal, există câteva secvențe în care perspectiva aparține personajelor. Acest lucru este un element de modernitate, purtând denumirea de \textbf{multiplicarea perspectivei narative}. De exemplu, descrierea străzii Antim și a casei lui Costache Giurgiuveanu, din începutul romanului, se face din perspectiva lui Felix.
\section{Structură. Subiect}
Romanul are o \textbf{structură simetrică} dată de relația de asemănare și de diferențiere dintre incipit și final. Replica lui Costache Giurgiuveanu, absurdă la început, \qu{Nu stă nimeni aici}, se încarcă în final de o tristețe existențială, întrucât casa va fi părăsită zece ani mai târziu.

\textbf{Expozițiunea} romanului este una tipic realistă, deoarece fixează acțiunea în timp și spațiu: \qu{Într-o seară de la începutul lui iulie 1909 ...}, pe strada Antim din București. Expozițiunea introduce și unul dintre personajele romanului, fiind vorba de Felix Sima, un tânăr de 18 ani, \qu{îmbrăcat în uniformă de licean}. Rămânând orfan, acesta vine la Costache Giurgiuveanu, unchiul și tutorele său legal, pentru a urma facultatea de medicină.

Este prezentă \textbf{tehnica focalizării}, constând în restrângerea treptată a cadrului, de la exterior la interior. În acest sens, din perspectiva lui Felix Sima este descrisă mai întâi strada, apoi casa lui moș Costache, fațada, interioarele și personajele.

\textbf{Tehnica realistă a detaliului} presupune descrieri amănunțite, care au rolul de a caracteriza indirect personajele. De exemplu, detaliile arhitectonice ale casei lui Costache Giurgiuveanu sugerează diferența între aparență și esență specifică burgheziei bucureștene, o categorie socială cu resurse materiale, dar fără fond cultural. De exemplu, îmbinarea unor stiluri incompatibile sugerează lipsa de stil, incultura; materialele ieftine evidențiază avariția; zidăria crăpată și scorojită sugerează delăsarea.

\textbf{Reuniunea familială} din salonul lui Giurgiuveanu devine un pretext narativ pentru prezentarea personajelor și pentru anticiparea conflictelor.

Planul narativ central este \textbf{istoria moștenirii lui Costache Giurgiuveanu}, plan dinamizat de un \textbf{conflict succesoral} între clanul Tulea și Otilia Mărculescu. Costache Giurgiuveanu amână înfierea Otiliei atât din avariție, cât și din cauza intrigilor familiei Tulea, care dorea să fie moștenitoarea unică a averii bătrânului.

Relevante pentru dezumanizarea burgheziei dominate de obsesia banului sunt episoadele care descriu reacțiile clanului Tulea în contextul îmbolnăvirii lui Costache Giurgiuveanu. La \textbf{primul atac de cord} al lui Giurgiuveanu, aceștia \qu{ocupă casa militărește}. Scena scoate în evidență tragicul și grotescul unei lumi insensibile, degradate moral. De exemplu, pentru a sublinia lipsa lor de empatie și imposibilitatea unei comunicări reale, autorul construiește un dialog inspirat din teatrul absurdului. Astfel, Aglae, fără pic de compasiune pentru fratele ei, vorbește despre o vagă boală proprie; Aurica, obsedată ca întotdeauna de căsătorie, vorbește despre norocul fetelor care se mărită; Stănică își amintește cu cinism despre veghea unui muribund.

După cel de-\textbf{al doilea atac de cord} al lui moș Costache, clanul Tulea începe să care din casă toate lucrurile de valoare. \textbf{Scena jafului nocturn} reliefează, încă o dată, lăcomia și insensibilitatea personajelor. Revenindu-și și acum, Giurgiuveanu îi încredințează lui Pascalopol o parte din banii de sub saltea, pentru a-i deschide Otiliei un cont în bancă. Scena este spionată de Stănică Rațiu, care îi sustrage lui Costache restul banilor, gest care îi este fatal bătrânului, provocându-i moartea.

\textbf{Destinul tânărului Felix} este marcat de \textbf{experiența inițiatică a iubirii}. \textbf{Idila adolescentină} dintre Felix și Otilia parcurge toate etapele și dilemele primei iubiri: geneza sentimentelor, exteriorizarea acestora, intensificarea lor și opțiunea rațională care pune capăt relației. Otilia îl părăsește pe Felix, plecând cu Pascalopol la Paris. Sfârșitul acestei povești de dragoste vorbește despre caracterul iluzoriu al libertății de a alege iubirea într-o lume în care totul este condiționat de factorul social, economic. Otilia înțelege forța acestui mecanism, renunțând la visul fericirii prin iubire. \textbf{Dragostea lui Pascalopol} implică generozitate, atât materială, cât și sufletească, deoarece el va renunța la Otilia, redându-i libertatea, când înțelege că fata nu mai este fericită.

\textbf{Epilogul} rezumă destinele personajelor după zece ani. De exemplu, aflăm despre Felix că a devenit profesor universitar la facultatea de medicină și că a făcut o alianță avantajoasă care i-a permis intrarea în lumea bună a capitalei. Felix află de la Stănică ultimele noutăți despre Otilia care a divorțat de Pascalopol, căsătorindu-se cu un conte.

Întâlnirea cu Pascalopol accentuează aura de mister a Otiliei. Pascalopol îi arată lui Felix o fotografie recentă a Otiliei. Acesta rămâne surprins, deoarece Otilia de care fusese îndrăgostit odinioară nu avea nimic în comun cu această doamnă marcată de \qu{un aer de platitudine}. Pascalopol concluzionează, afirmând că pentru el Otilia rămâne o \textbf{enigmă}. Felix generalizează, afirmând că destinul însuși este o enigmă.
\section{Concluzie}
În concluzie, \qu{Enigma Otiliei} pare a fi o demonstrație a unui program artistic, \textbf{caracterul balzacian} fiind surprins și de Pompiliu Constantinescu: \qu{Domnul Călinescu nu s-a preocupat de nicio teorie la modă [...] A procedat clasic, după metoda balzaciană}.
\part*{Caracterizare}
\setcounter{section}{0}
\section{Introducere}
Otilia Mărculescu, protagonista romanului \qu{Enigma Otiliei} de George Călinescu, simbolizează \textbf{feminitatea în devenire}.

Pentru majoritatea personajelor din roman George Călinescu folosește tehnici de caracterizare realiste, balzaciene. \textbf{Tehnica realistă a detaliului} constă în descrieri amănunțite care au rolul de a caracteriza indirect personajele, indicând trăsături de caracter, de personalitate.

\textbf{În cazul Otiliei}, autorul folosește o tehnică modernă de caracterizare, și anume \textbf{pluriperspectivismul}. Conform acestei tehnici, personajul este perceput în mod diferit de către celelalte personaje ale romanului, rezultând un portret alcătuit din trăsături diferite.

Personajele romanului sunt \textbf{de factură realistă}, încadrându-se în \textbf{tipologii}, pe baza trăsăturii dominante de caracter. De exemplu, Aglae Tulea este \qu{baba absolută, fără cusur în rău}, Aurica este fata bătrână, Stănică Rațiu este parvenitul.

\textbf{Un element de modernitate} este \textbf{ambiguitatea personajelor} care nu se încadrează într-o singură schemă tipologică. De exemplu, moș Costache nu este un avar dezumanizat, el nutrește o iubire paternă sinceră pentru Otilia. El este o combinație între două caractere balzaciene: \textbf{avarul} și \textbf{tatăl}.
\section{Cuprinsul}
Otilia Mărculescu, o tânără de 18--19 ani, este fiica celei de a doua soții a lui Costache Giurgiuveanu. Aceasta îi lăsase, pe lângă avere, datoria de a o crește pe Otilia. Costache o iubește sincer ca pe fiica lui, dar avariția și intrigile clanului Tulea îl împiedică să o adopte legal.

\textbf{Portretul fizic} al personajului este conturat încă de la începutul romanului, din perspectiva lui Felix: \qu{Fața măslinie, cu nasul mic și ochii foarte albaștri [...] Trupul subțiratic cu oase delicate de ogar, de un stil perfect}. Secvența descriptivă sugerează \textbf{suplețea} și \textbf{delicatețea} fetei.

\textbf{Tehnica balzaciană a detaliului} este folosită și în cazul Otiliei. De exemplu, oglinzile mobile din camera Otiliei sugerează \textbf{firea versatilă și imprevizibilă} a fetei. Amestecul tineresc al lucrurilor denotă firea exuberantă și dezinvoltă. Despre amestecul de rochii, pălării, pantofi, jurnale de modă, partituri muzicale, naratorul spune că reprezintă \qu{ascunzișul ei feminin}.

Otilia este caracterizată prin \textbf{tehnica modernă a pluriperspectivismului}, care accentuează \textbf{aura de mister, ambiguitatea} personajului. De exemplu, pentru moș Costache, Otilia este \qu{fe-fetița} lui, pentru Felix este \qu{o fată admirabilă, superioară}, dar pe care n-o poate înțelege; pentru Pascalopol este \qu{o femeie în devenire, cu un temperament de artistă}; pentru Stănică Rațiu este \qu{o fată cu spirit practic, care știe ce vrea și cum să se descurce în viață}; pentru Aglae și Aurica Tulea este \qu{o stricată, o dezmățată}, \qu{o fată fără căpătâi și fără părinți}.

Cu o \textbf{intuiție deosebită}, Otilia se comportă adecvat fiecărei situații și fiecărei persoane: este copilăroasă și nebunatică cu Pascalopol, tandră și protectoare cu Felix, grijulie cu moș Costache, ironică și distantă față de clanul Tulea.

\textbf{Otilia este caracterizată indirect prin relația cu celelalte personaje}. \textbf{În relație cu Felix}, Otilia se ghidează după reguli raționale. Astfel, ea îi explică faptul că o iubire devoratoare de tinerețe poate reprezenta un obstacol în calea dezvoltării lui profesionale. Otilia înțelege că, în lumea în care trăiește, iubirea este condiționată de factorul social, economic. Felix nu înțelege că Otilia are nevoie de o iubire matură care să îi ofere siguranța materială, libertatea de mișcare și de acțiune. În consecință, Otilia va renunța la visul fericirii prin iubire, căsătorindu-se cu Pascalopol.

\textbf{Pascalopol} are o atitudine oscilantă față de Otilia, pendulând între dragostea paternă și cea virilă. Matur, educat, el trăiește în preajma ei pentru a-i oferi confortul de care are nevoie. Dragostea lui Pascalopol implică generozitate, deoarece el va renunța la Otilia, redându-i libertatea, când înțelege că nu mai este fericită.

Otilia însăși recunoaște că este o \textbf{fire dificilă}, autocaracterizându-se în felul următor: \qu{Sunt foarte capricioasă, vreau să fiu liberă! ... mă plictisesc repede, sufăr când sunt contrariată}.

Călătoria la Paris, cu Pascalopol, o maturizează pe Otilia, care devine \textbf{mai sigură, mai conștientă de ea însăși}. Schimbarea aceasta o percepe Felix, pe care seriozitatea fetei \qu{îl paraliza}.

Fotografia pe care i-o arată Pascalopol în finalul romanului îl surprinde pe Felix. Doamna din fotografie, cu \qu{un aer de platitudine feminină} nu avea nimic în comun cu Otilia de care se îndrăgostise odinioară.

Fotografiile Otiliei, din momente diferite ale romanului sugerează \textbf{mobilitatea personajului} surprins în \textbf{ipostaze diferite ale feminității sale}: copilă, adolescentă, femeie.
\section{Concluzie}
În concluzie, Otilia reprezintă un personaj bovaric care trăiește drama trecerii timpului, deoarece \qu{o femeie se bucură de viața adevărată doar câțiva ani}, așa cum i se confesează ea lui Felix.
\end{document}
