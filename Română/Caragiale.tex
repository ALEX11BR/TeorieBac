\documentclass{article}
\usepackage{enumitem}
\usepackage{indentfirst}
\usepackage{listings}
\usepackage{graphicx}
\usepackage{caption}
\usepackage[romanian]{babel}
\usepackage{verse}
\usepackage[a4paper,portrait,margin=1in]{geometry}
\usepackage{multicol}
\pagenumbering{gobble}

\renewcommand\thesection{\arabic{section}.}
\renewcommand\thesubsection{\thesection\arabic{subsection}.}

\newcommand{\qu}[1]{„\emph{#1}”}

\title{O scrisoare pierdută}
\author{de I. L. Caragiale}
\date{}
\begin{document}
\maketitle

\part*{Comentariu}
\section{Încadrarea în context}
Referindu-se la comediile lui I. L. Caragiale, Eugen Ionescu este de părere că acestea aduc pe scenă o lume de antieroi, \qu{personajele fiind niște exemplare umane în așa măsură de degradate, încât nu ne lasă nicio speranță de salvare}.

În opera lui Caragiale se conturează \textbf{două viziuni despre lume}: una \textsl{\textbf{comică și critică}} în comedii și în schițe și una \textsl{\textbf{gravă, serioasă}} în nuvele și în drama \qu{Năpasta}. Observator atent al societății românești din vremea lui, Caragiale se transformă într-un \textbf{scriitor realist și moralizator}.

Comedia \qu{O scrisoare pierdută} s-a jucat pentru prima dată pe scena Teatrului Național din București în anul 1884. Este o \textbf{comedie de moravuri sociale, politice și familiale}, deoarece satirizează defecte precum: corupția, demagogia, trădarea, minciuna. Comedia este specia genului dramatic în care sunt criticate aspecte ale vieții sociale prin intermediul tipurilor de comic. Prin urmare, în comedii, \textbf{râsul are un rol moralizator}, deoarece autorul intenționează îndreptarea defectelor prin intermediul tipurilor de comic.
\section{Titlul}
Titlul piesei \textbf{anticipează intriga}, sugerând faptul că acea scrisoare de amor este doar unul dintre mijloacele de șantaj în lupta politică. Scrisoarea de dragoste, trecută prin mai multe mâini, devine un \textbf{simbol al corupției și al compromisului}.
\section{Teme}
Tema dominantă a comediei o constituie \textbf{prezentarea vieții social-politice} dintr-un oraș de provincie, în preajma alegerilor pentru Camera Deputaților.
\section{Structură. Subiect}
\textbf{Cronotopul} este precizat în \textbf{incipit}, într-una dintre notațiile autorului: \qu{În capitala unui județ de munte, în zilele noastre}. Nu se precizează numele orașului tocmai pantru a putea generaliza situația, pentru a putea plasa acțiunea în orice parte a țării.

\textbf{Compozițional}, comedia este structurată pe \textbf{patru acte}, în care se amplifică treptat tensiunea provocată de evenimentul politic și de pierderea scrisorii.
\subsection{Expozițiunea}
Expozițiunea îl prezintă pe Ștefan Tipătescu, prefectul județului, răsfoind ziarul lui Cațavencu, \qu{Răcnetul Carpaților}. Pristanda, polițistul și omul de încredere al puterii, îi raportează prefectului misiunea pe care o avusese noaptea trecută. Trecând pe lângă casa lui Cațavencu, polițistul află că acesta se află în posesia unui document foarte important, care îi va aduce funcția de deputat.
\subsection{Intriga}
Intriga o constituie pierderea scrisorii de amor adresată Zoei de către Tipătescu, ceea ce va declanșa desfășurarea întregii acțiuni.
\subsection{Desfășurarea acțiunii}
Cațavencu îi cheamă la redacția ziarului pe Trahanache, apoi pe Zoe, pe care îi amenință cu publicarea scrisorii, dacă nu îi susțin candidatura. Trahanache nu crede în autenticitatea documentului, considerându-l \qu{o plastografie}.

Își fac apariția Farfuridi și Brânzovenescu care îi bănuiesc pe aliații lor politici de trădare, văzându-i în vizită la Nae Cațavencu.

\textbf{Conflictul principal al comediei} îmbracă forma unui conflict politic, dar confruntarea nu se produce între doctrine politice diferite, ci între două grupări diferite ale aceluiași partid de guvernământ. \textbf{Conflictul se declanșează} odată cu amenințarea lui Cațavencu de a publica scrisoarea de amor și evoluează treptat pe măsură ce îi șantajează pe Trahanache și pe Zoe. Caragiale folosește \textbf{tehnica amplificării treptate a conflictului} prin situații neașteptate: intrările și ieșirile repetate ale Cetățeanului Turmentat, răsturnări neașteptate de situație.

Arestat ilegal din ordinul prefectului, Cațavencu este eliberat la decizia Zoei și invitat \textbf{la negocieri}. Acesta refuză funcțiile și moșiile propuse de Tipătescu, dorind doar \qu{deputăția}. La insistențele Zoei, care se folosește de toate armele seducției feminine, Tipătescu acceptă să sprijine candidatura lui Cațavencu.

\textbf{Actul al III-lea} al comediei este actul discursurilor electorale. Este prezent \textbf{comicul de caracter} care constă în evidențierea trăsăturilor dominante de caracter ale celor doi candidați. Farfuridi este \textsl{\textbf{prostul fudul}}, iar Nae Cațavencu este \textsl{\textbf{demagogul}} pentru care noțiuni precum \qu{țară}, \qu{progres}, \qu{patriotism} sunt doar lozinci electorale. Tot în acest act este prezent \textbf{comicul de limbaj} care constă în diverse abateri de la normă: \textsl{\textbf{confuzii semantice}} cum ar fi \qu{capitaliști} care, pentru Nae Cațavencu, înseamnă \qu{locuitori ai capitalei}; \textsl{\textbf{lipsa de coerență a discursurilor}}, nonsensuri precum fraza rostită de Cațavencu: \qu{Industria română e admirabilă, e sublimă, [...] dar lipsește cu desăvârșire}.

\textbf{Conflictul} atinge intensitatea maximă în momentul în care este anunțat candidatul câștigător: \mbox{Agamemnon Dandanache}. Numele câștigătorului este primit pe un bilețel de la sediul central al partidului, declanșând un adevărat haos, între taberele de alegători, când este pronunțat de Trahanache. \textbf{Haosul} poate fi considerat o \textbf{metaforă scenică} pentru o lume debusolată care nu distinge între aparență și esență.

Dandanache se dovedește a fi \qu{mai canalie decât toți}, deoarece obține mandatul de deputat tot printr-un șantaj cu o scrisoare de amor. Trahanache găsește o modalitate de a răspunde la șantajul lui Cațavencu tot prin șantaj, deoarece găsește o poliță falsificată, prin care avocatul a sustras bani de la societatea pe care o conducea.

Cetățeanul Turmentat găsește pălăria lui Cațavencu în căptușeala căreia era ascunsă scrisoarea de amor. \textbf{Conflictul se stinge} în momentul în care Cetățeanul Turmentat înapoiază scrisoarea \qu{andrisantului}, adică Zoei. Fără obiectul șantajului, Cațavencu devine umil, acceptând chiar să conducă festivitatea organizată în cinstea câștigătorului.
\subsection{Deznodământul}
Deznodământul pune în evidență \textbf{comicul de situație} care constă în răsturnări de situație, în evoluția surprinzătoare a lui Cațavencu de la \textbf{stăpân pe situație}, așa cum părea inițial, la \textbf{slugarnic}, gata să execute ordinele Zoei.

Atmosfera din finalul comediei este una veselă, adversarii politici se împacă, accentuând \textbf{nota de farsă electorală} a comediei.
\section{Concluzie}
În concluzie, în comedia \qu{O scrisoare pierdută} este demonstrată \textbf{vocația de scriitor realist} a lui Caragiale, evidențiată de înfățișarea unor tipuri sociale contemporane și a unor situații concrete.

\part*{Caracterizare}
\setcounter{section}{0}
\section{Introducere}
\textbf{Eugen Ionescu} afirma despre comediile lui Caragiale că aduc pe scenă o lume de antieroi, \qu{personajele comediei fiind în așa măsură de degradate încât nu ne lasă nicio speranță de salvare}.

\textbf{Personajele de comedie} sunt, în genere, tipuri umane mediocre, degradate moral, pline de vicii. În cazul acesta, râsul are un \textbf{rol moralizator}, deoarece autorul intenționează corectarea defectelor, prin intermediul tipurilor de comic.

Personajele comediilor lui Caragiale sunt \textbf{personaje realiste} care se încadrează în diverse tipologii, pe baza trăsăturilor dominante de caracter. Criticul literar \textbf{Pompiliu Constantinescu} identifică următoarele tipologii sociale, umane în comedia \qu{O scrisoare pierdută}: Zaharia Trahanache -- încornoratul; Ștefan Tipătescu -- junele prim, amorezul, Don Juanul; Zoe -- cocheta; Pristanda -- funcționarul umil, servilul.
\section{Cuprinsul}
Nae Cațavencu este caracterizat atât direct, cât și indirect prin fapte, comportament și relația cu celelalte personaje ale comediei.

Cațavencu este \textbf{caracterizat indirect} prin statut, care este detaliat la începutul comediei, în lista de personaje. Astfel, Cațavencu este avocat, directorul ziarului \qu{Răcnetul Carpaților}, liderul opoziției politice din județ și președintele fondator al societății \qu{Aurora Economică Română}.

Personajul este \textbf{caracterizat indirect prin nume}. În comediile lui Caragiale, numele personajelor sugerează o trăsătură de caracter a personajelor. În cazul lui Cațavencu, etimologia numelui poate fi interpretată în două moduri: fie este derivat de la substantivul \qu{cață} care înseamnă \qu{femeie bârfitoare}, sugerând \textbf{demagogia} personajului; fie provine de la substantivul \qu{cațaveică} ce înseamnă \qu{haină cu două fețe}, sugerând dualitatea, ipocrizia personajului.

\textbf{Ca lider al opoziției}, Cațavencu este \textbf{lipsit de principii și de demnitate}. El sustrage scrisoarea de amor de la Cetățeanul Turmentat și o transformă în instrument de șantaj politic. În sensul acesta, Cațavencu îi cheamă la redacția ziarului pe Trahanache, apoi pe Zoe, pe care îi amenință cu publicarea scrisorii dacă nu îi susțin candidatura.

Personajul se conduce după deviza \qu{Scopul scuză mijloacele}, aparținând lui Machiavelli. De aici reiese faptul că este \textbf{lipsit de principii} și \textbf{oportunist}.

Nae Cațavencu este \textbf{caracterizat indirect prin fapte}: falsifică o poliță pentru a sustrage niște bani, sustrage scrisoarea Cetățeanului Turmentat, îi amenință și șantajează pe liderii politici ai județului, este gata să treacă în orice tabără, să încalce orice lege juridică sau morală.

O altă trăsătură caracteristică este \textbf{duplicitatea}, \textbf{ipocrizia}. Nae Cațavencu este \textbf{arogant} și \textbf{agresiv}, amenințând cu publicarea scrisorii, pentru ca ulterior, după ce pierde scrisoarea, să devină \textbf{umil} și \textbf{lingușitor}. \textbf{Evoluția surprinzătoare} a personajului, de la aparent câștigător la perdant, pune în evidență comicul de situație.

Avocatul este \textbf{inflexibil} și \textbf{arogant} în secvența negocierii. Cațavencu refuză moșiile și funcțiile propuse de Tipătescu, nedorind nimic altceva decât \qu{deputăția}. Secvența pune în evidență, de asemenea, \textbf{ambiția} și \textbf{dorința} nemărginită a personajului \textbf{de a parveni}.

\textbf{Discursul electoral} al lui Cațavencu pune în evidență \textbf{lipsa de educație}, \textbf{demagogia} si \textbf{capacitatea de disimulare} a personajului. Incultura, lipsa de educație reies din abaterile de la norma literară, discursul său fiind plin de erori semantice, gramaticale, logice. De exemplu, Cațavencu face mai multe confuzii semantice, folosind termenul de \qu{capitaliști} cu sensul de \qu{locuitori ai capitalei}. Discursul său este lipsit de coerență din cauza unor nonsensuri precum \qu{Industria română e admirabilă, e sublimă, [...] dar lipsește cu desăvârșire}.

Cațavencu este un \textbf{actor desăvârșit}, el știind să simuleze emoția pentru a influența electoratul. Pentru Nae Cațavencu, noțiuni precum \qu{țară}, \qu{progres}, \qu{popor} reprezintă simple lozinci electorale.

Personajul are \textbf{capacitatea de a se adapta la orice situație}, astfel el acceptă în final să conducă festivitatea organizată în cinstea lui Dandanache.

Imaginea finală a personajului este aceea a unui orator amețit care se împacă cu adversarii lui politici, accentuând nota de farsă electorală a comediei. 
\section{Concluzie}
În concluzie, Nae Cațavencu este un personaj tipic de comedie, degradat moral, plin de vicii, o mască ce acoperă vidul interior.
\end{document}
