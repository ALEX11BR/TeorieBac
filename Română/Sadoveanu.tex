\documentclass{article}
\usepackage{enumitem}
\usepackage{indentfirst}
\usepackage{listings}
\usepackage{graphicx}
\usepackage{caption}
\usepackage[romanian]{babel}
\usepackage{verse}
\usepackage[a4paper,portrait,margin=1in]{geometry}
\usepackage{multicol}
\pagenumbering{gobble}

\renewcommand\thesection{\arabic{section}.}
\renewcommand\thesubsection{\thesection\arabic{subsection}.}

\newcommand{\qu}[1]{„\emph{#1}”}

\title{Baltagul}
\author{de Mihail Sadoveanu}
\date{}
\begin{document}
\maketitle

\part*{Comentariu}
\section{Încadrarea în context}
Publicat pentru prima dată în 1930, \qu{Baltagul} de Mihail Sadoveanu ilustrează \textbf{formula tradițională a romanului realist, de observație socială și de problematică morală}.

Complexitatea romanului este dată de \textbf{diversitatea cheilor de lectură} în care acesta poate fi interpretat. De exemplu, \qu{Baltagul} poate fi citit ca \textbf{roman al transhumanței} care zugrăvește o civilizație pastorală, arhaică; \textbf{roman inițiatic} care prezintă inițierea lui Gheorghiță în viață; \textbf{roman al familiei} și chiar un \textbf{roman cu intrigă polițistă}.

Printre \textbf{caracteristicile tradiționalismului} întâlnim: valorificarea tradițiilor și a obiceiurilor de nuntă, botez, înmormântare; satul ca spațiu de desfășurare a acțiunii; cronologia întâmplărilor; obiectivitatea și finalul închis.
\section{Titlul}
În cazul de față, titlul pune întregul univers al romanului sub \textbf{semnul dualității}. Baltagul, un topor cu două tăișuri, este în același timp \textbf{armă a crimei} și \textbf{unealtă a dreptății}.
\section{Teme}
Tema \textbf{vieții} și a \textbf{morții} și cea a \textbf{căutării adevărului} sunt dominante în roman și sunt detaliate prin \textbf{motivul călătoriei}. Pentru Vitoria, călătoria este una \textbf{explorativă} care are ca scop aflarea adevărului despre absența prelungită a lui Nechifor. Pentru Gheorghiță, călătoria este una \textbf{inițiatică}, având ca efect maturizarea personajului, în final preluând statutul de stâlp al familiei.
\section{Perspectivă narativă. Narator}
Romanul este scris la persoana a III-a, având o \textbf{perspectivă narativă obiectivă}. Punctul de vedere dominant aparține naratorului obiectiv și omniscient care dirijează evoluția personajelor \qu{ca un regizor impersonal}, așa cum afirmă Nicolae Manolescu.
\section{Structură. Subiect}
Subiectul romanului poate fi structurat pe următoarele \textbf{planuri narative}: \textbf{planul existenței individuale și familiale} care urmărește călătoria explorativă a Vitoriei; \textbf{planul existenței comunității de păstori}, un plan monografic care surprinde existența unei lumi arhaice; \textbf{planul mitic și simbolic} la care individul de raportează prin credință și prin gândire magică, adică prin superstiții, semne, vise.

Acțiunea se desfășoară în \textbf{spații reale} și urmărește evoluția personajelor într-un \textbf{timp real}. Personajele sadoveniene sunt surprinse într-un spațiu real, de la Măgura Tarcăului până în ținutul Dornelor, și într-o durată reală, care acoperă aproximativ o jumătate de an: din toamnă până în primăvară. Spațiul real și timpul obiectiv sunt dublate de \textbf{un spațiu simbolic} (traseul labirintic și spațiul oniric al visului) și de \textbf{un timp mitic}, din legenda cu care se deschide romanul.
\subsection{Incipitul}
Incipitul romanului rezumă o \textbf{legendă} care are următoarele \textbf{roluri}: schițează portretul personajului colectiv -- ciobanii, cărora Dumnezeu le-a dat \qu{o inimă ușoară}; introduce personajul absent al romanului: \qu{Povestea asta o spunea uneori Nechifor Lipan la cumetrii și nunți, la care era nelipsit}.
\subsection{Expozițiunea}
În expozițiune este descrisă existența comunității de păstori din Măgura Tarcăului și a familiei Lipan. Nechifor Lipan, capul familiei, plecase să vândă și să cumpere oi. Întârzierea lui o neliniștește pe Vitoria care îl cheamă pe Gheorghiță, fiul ei, de la stână.

Neliniștea femeii este amplificată de unele \textbf{semne ale naturii} și de un \textbf{vis premonitoriu}, în care Nechifor, întors cu spatele la ea și cu fața spre asfințit, traversează călare o apă neagră.
\subsection{Intriga}
Intriga romanului este marcată de hotărârea Vitoriei de a reconstitui, împreună cu fiul ei, traseul parcurs de Nechifor.
\subsection{Desfășurarea acțiunii}
Desfășurarea acțiunii cuprinde pregătirile pentru drum ale femeii, care încep cu o călătorie la Piatra pentru a anunța autorităților dispariția lui Lipan.

\textbf{Pregătirile administrative sunt dublate de cele spirituale}. De exemplu, Vitoria lasă gospodăria în grija unui argat; pe Minodora o duce la mănăstire, în grija unei maici; se purifică spiritual prin post și rugăciune; sfințește baltagul lui Gheorghiță, pentru a avea protecția divinității pe tot parcursul drumului. Vitoria îmbină credința cu gândirea magică, sfătuindu-se atât cu preotul, cât și cu vrăjitoarea satului.

Împreună cu fiul ei, Vitoria se oprește din loc în loc pentru a stânge informații despre \qu{omul cu căciulă brumărie}. Traseul reconstituit conține multe repere spațiale, putând fi asociat cu \textbf{motivul labirintului}, la capătul căruia se află adevărul despre dispariția lui Lipan. În continuare, în drumul lor, Vitoria și Gheorghiță se opresc și respectă tradițiile de botez și de nuntă. Acestea au un \textbf{rol anticipativ}, anunțând un al treilea eveniment crucial din existența individului -- moartea.

\textbf{La Dorna}, Vitoria află că soțul ei a cumpărat câteva sute de oi. \textbf{La Sabasa}, turma de oi era însoțită de Nechifor și de alți doi ciobani din partea locului: Calistrat Bogza și Ilie Cuțui. \textbf{La Suha}, turma era însoțită doar de cei doi ciobani din partea locului.

\textbf{O secvență emoționantă} este găsirea câinelui lui Nechifor, Lupu, care îi conduce la rămășițele lui Lipan din râpa de sub Crucea Talienilor. \textbf{Veghea osemintelor} tatălui reprezintă o probă inițiatică pentru Gheorghiță, o înfruntare a temerilor. În timpul acesta, Vitoria pleacă să anunțe autoritățile.

\textbf{Respectul față de tradiții} și dragostea față de soțul răposat o determină pe Vitoria să organizeze praznicul și slujba de înmormântare. Cu abilitate și disimulare, Vitoria regizează scena demascării vinovaților, ceea ce îl determină pe George Călinescu să o considere un adevărat \qu{Hamlet feminin}. Sub pretextul dialogului purtat în vis cu Nechifor, Vitoria reconstituie pas cu pas scena crimei. Constrâns de autorități și de familie, Calistrat recunoaște crima. Secvența în care Gheorghiță îl lovește cu baltagul pe ucigașul tatălui său este o \textbf{valorificare a mitului crengii de aur}. Baltagul, echivalentul crengii de aur, facilitează transferul de cunoștințe și de experiență de viață de la tată la fiu, acesta din urmă devenind noul stâlp al familiei.
\subsection{Deznodământul}
După ce dreptatea a fost restabilită, totul reintră în tiparul vieții de la munte. Astfel, deznodământul o prezintă pe Vitoria care stabilește împreună cu Gheorghiță sarcinile pentru ziua următoare.
\section{Concluzie}
În concluzie, \qu{Baltagul} este un text reprezentativ pentru tradiționalismul interbelic, deoarece este un roman care \qu{preferă psihologiei fapta, iar analizei epicul}, așa cum a afirmat Nicolae Manolescu.

\part*{Caracterizare}
\setcounter{section}{0}
\section{Introducere}
Vitoria Lipan este protagonista romanului \qu{Baltagul} de Mihail Sadoveanu, despre care criticul literar D. P. Perpessicius a afirmat că este \qu{romanul unui suflet de munteancă}.

\textbf{Personajele sadoveniene} au o înțelepciune venită din adâncurile vremurilor și o legătură secretă cu natura.

Ca \textbf{eroină de roman tradiționalist}, Vitoria Lipan este surprinsă în relație cu mediul, cu acea comunitate de păstori din care face parte.

Chiar dacă luăm în considerare opinia lui George Călinescu, și anume că Vitoria Lipan este o \qu{exponentă a speței}, ea depășește categoria generalului și devine personaj individualizat prin limbaj, fapte, comportament.
\section{Cuprinsul}
\textbf{Portretul fizic}, prezentat succint la începutul romanului, este unul general: o femeie de 40 de ani, cu părul lung și ochii căprui. Un detaliu, însă, atrage atenția, și anume privirea care era dusă departe. Aceasta sugerează \textbf{neliniștea sufletească} a protagonistei, generată de absența prelungită a lui Nechifor.

\textbf{Portretul moral} al protagonistei se conturează prin însumarea mai multor \textbf{ipostaze}: femeie de la munte, soție, mamă, erou justițiar.

Ca femeie de la munte, Vitoria împrumută ceva din tăria stâncilor pe care comunitatea de păstori își duce traiul. \textbf{Femeie puternică, deprinsă cu greutățile vieții}, Vitoria este o \textbf{gospodină harnică} și pricepută care preia conducerea gospodăriei în absența soțului. Ea știe să se tocmească cu negustorii și unde să-și vândă mai bine produsele.

Vitoria are un \textbf{simț practic} deosebit, plecând în căutarea lui Nechifor după ce lasă treburile administrative în grija unor oameni de încredere. \textbf{Prevăzătoare}, ea își duce banii la preotul satului pentru a nu fi prădată, comandă pentru Gheorghiță un baltag și ia cu sine o pușcă pe care să o folosească la nevoie.

\textbf{Vitoria îmbină credința cu gândirea magică}, respectând atât obiceiurile creștine, cât și legile nescrise ale comunității. De exemplu, ea nu pleacă la drum până nu se consultă atât cu preotul, cât și cu vrăjitoarea satului, se purifică spiritual prin post și rugăciune, sfințește baltagul fiului, organizează praznicul și slujba de înmormântare pentru răposat. În același timp, \textbf{Vitoria este superstițioasă}, descifrând visele și semnele naturii. De exemplu, bănuiala că soțul ei este mort este întărită de visul premonitoriu în care Lipan, întors cu fața către apus, traversează călare o apă neagră.

Ca \textbf{soție devotată}, dragostea pe care i-o poartă lui Lipan o face să fie \textbf{neliniștită}, împovărată de gânduri. Bănuiala că el este mort o macină \qu{ca un vierme neadormit}. În dragostea ei, există certitudinea că este iubită și că Nechifor se va întoarce întotdeauna la ea \qu{ca la apa cea bună}.

\textbf{Ca mamă}, Vitoria îi educă pe cei doi copii în spiritul respectării tradiției. Minodora este inițiată \textbf{cu severitate} în datoriile ei zilnice de viitoare soție. Pe durata absenței mamei de acasă, fiica este dusă la mănăstire. Educația lui Gheorghiță se realizează cu alte mijloace, în aparență mai blânde și mai permisive. Vitoria vede în el oglinda bărbatului ei, îl ia ca sprijin în călătorie, și, în final, îi atribuie statutul de stâlp al familiei.

Pe parcursul călătoriei, Vitoria dă dovadă de \textbf{luciditate} și \textbf{înțelepciune}, făcând traseul numai pe timp de zi, sub lumina protectoare a soarelui. Se oprește din loc în loc pentru a strânge informații pe care, cu o \textbf{logică impecabilă}, le organizează pentru a reconstitui traseul parcurs de Nechifor.

Călătoria aceasta o pune pe Vitoria în relație cu oameni necunoscuți. În situația aceasta, ea demonstrează \textbf{o fină cunoaștere a psihologiei umane}, intuind firea ascunsă a oamenilor. Această calitate este remarcată și de Gheorghiță, care o \textbf{caracterizează direct} pe mama sa în felul următor: \qu{Mama trebuie să fie fermecătoare, cunoaște gândul nerostit al omului}.

Vitoria știe că are un \textbf{dublu mandat}: \textsl{\textbf{cel uman}}, care presupune identificarea, demascarea și pedepsirea ucigașilor, și \textsl{\textbf{cel divin}}, care presupune înfăptuirea ritualului funerar care asigură odihna veșnică a răposatului.

În metoda cu care cercetează crima, se întrezărește un amestec de \textbf{intuiție} și \textbf{capacitate de disimulare}. Eroina știe instinctiv cum să interogheze martorii, cum să reconstituie faptele, motiv pentru care George Călinescu o consideră un adevărat \qu{Hamlet feminin}. Pedepsirea ucigașilor soțului ei finalizează \textbf{mandatul de justițiar} al eroinei.

După ce datoria este împlinită, se poate reintra în ordinea lumii. Vitoria și fiul ei pleacă spre Măgura Tarcăului pentru a lua lucrurile de unde le au lăsat.
\section{Concluzie}
În concluzie, cu un destin mitic ca al măicuței bătrâne din \qu{Miorița}, Vitoria Lipan rămâne unul dintre cele mai frumoase chipuri feminine din literatura română, simbol al iubirii care depășește încercările destinului.
\end{document}
