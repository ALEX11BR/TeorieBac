\documentclass{article}
\usepackage{enumitem}
\usepackage{indentfirst}
\usepackage{listings}
\usepackage{graphicx}
\usepackage{caption}
\usepackage[romanian]{babel}
\usepackage{verse}
\usepackage[a4paper,portrait,margin=1in]{geometry}
\usepackage{multicol}
\pagenumbering{gobble}

\renewcommand\thesection{\arabic{section}.}
\renewcommand\thesubsection{\thesection\arabic{subsection}.}

\newcommand{\qu}[1]{„\emph{#1}”}

\title{Ion}
\author{de Liviu Rebreanu}
\date{}
\begin{document}
\maketitle

\part*{Comentariu}
\section{Încadrarea în context}
Romanul \qu{Ion} de Liviu Rebreanu este publicat pentru prima dată în 1920, într-o perioadă când în literatura română circulă două tipare narative: \textbf{romanul tradiționalist} și \textbf{romanul modern}.

Romanul lui Rebreanu aparține \textbf{tradiționalismului interbelic} prin următoarele trăsături: valorificarea tradițiilor locale (hora duminicală, datini de botez, nuntă, înmormântare), acțiunea se desfășoară în mediul rural, tematica este una de factură socială, acțiunea este una cronologică, perspectiva narativă este obiectivă, iar finalul este unul închis prin soluționarea conflictelor.

Romanul \qu{Ion} se distanțează de rețeta de creație tradiționalistă prin câteva \textbf{accente antisămănătoriste}: protagonistul nu este țăranul imaculat din punct de vedere moral, așa cum preferă tradiționaliștii; satul din romanul lui Rebreanu nu este o lume idilică, ci una marcată de conflicte între diferite tipologii de țărani; nunta nu este bazată pe sentimente autentice, ci pe interese economice, sociale, fiind văzută ca un contract.
\section{Titlul}
În cazul de față, titlul anticipează protagonistul romanului. Numele său este unul neaoș, autohton, sugerând categoria socială a țărănimii, din care protagonistul face parte.
\section{Teme}
\textbf{Tema dominantă} a romanului este de factură socială și vizează încercările disperate ale unui țăran sărac de a-și depăși statutul social. Este detaliată legătura țăranului cu pământul.

Criticul literar Nicolae Manolescu remarcă faptul că \textbf{destinul prestabilit} poate funcționa ca temă secundară a romanului. În ciuda tuturor eforturilor lui Ion de a intra în posesia pământurilor lui Vasile Baciu, acesta le pierde în final. Destinul îl amendează pentru exces, lăcomie, abaterile de la normele morale.
\section{Perspectivă narativă. Narator}
Fiind un roman tradiționalist, este scris la persoana a III-a, având o \textbf{perspectivă narativă obiectivă}. Punctul de vedere dominant îi aparține naratorului obiectiv, omniprezent, omniscient. Omnisciența acestui narator reiese din semnele și secvențele cu rol anticipativ.
\section{Structură. Subiect}
Romanul are o \textbf{structură simetrică}, dată de asemănarea și diferențele dintre incipit și final. În ambele secvențe este descris drumul care intră, respectiv iese din satul Pripas. Drumul poate fi o \textbf{metaforă a ficțiunii}, care-l introduce pe cititor, apoi îl scoate din universul ficțional al romanului. Drumul acesta este unul șerpuitor, sugerând oscilarea lui Ion între cele două glasuri.

Atât în incipit, cât și în final este descrisă o cruce de lemn. În incipit, crucea este \qu{strâmbă}, iar Hristosul are \qu{fața spălăcită de ploi}, \textbf{detalii anticipative} care îl avertizează pe cititor că va cunoaște o lume desacralizată. În final, pe aceeași cruce de lemn, se proiectează o rază de soare ca simbol al speranței într-o lume mai bună, după ce răul a fost înlăturat.

Romanul este structurat pe două părți care corespund vocilor interioare ale protagonistului: \qu{Glasul pământului} și \qu{Glasul iubirii}.

O scenă relevantă este \textbf{hora duminicală} din curtea Todosiei, mama Floricăi. Nicolae Manolescu o numește \qu{o horă a soartei} deoarece aici se decid destinele mai multor personaje. Scena horei evidențiază \textbf{stratificarea socială} din satul Pripas. De exemplu, primarul discută cu chiaburii satului, familia învățătorului Herdelea analizează evenimentul de pe margine, iar Alexandru Glanetașu stă umil, \qu{ca un câine la ușa bucătăriei}.

La horă, Ion face primul pas și o invită pe Ana la dans. Fiecare pas al protagonistului este atent cântărit, Ion oscilând între Florica, frumoasă, dar săracă; și Ana, urâțică, dar \qu{cu locuri și case și vite multe}.

Tot la horă izbucnește \textbf{conflictul dintre Ion și Vasile Baciu}, tatăl Anei. Acesta se înfurie când află că Ana s-a retras cu Ion, căruia îi intuiește planul ascuns. Insultele lui Vasile Baciu atestă felul în care este perceput țăranul sărac: \qu{sărăntocule}, \qu{tâlharule}, \qu{hoțule}.

După horă, la cârciumă, are loc un alt \textbf{conflict între Ion și George Bulbuc}, un țăran înstărit pe care Vasile Baciu îl preferă ca ginere. Pretextul este plata lăutarilor, dar adevăratul motiv este competiția dintre cei doi pretendenți ai Anei.

După incidentul de la cârciumă, Ion este mustrat de preotul Belciug, după slujbă, în fața întregii comunități: \qu{Ești un stricat și un bătăuș și un om de nimic!}. Umilit, Ion se răzbună, rugându-l pe învățător să-i scrie o plângere adresată Ministerului Justiției din care să reiasă că Ion este victima preotului și a altor țărani din sat.

\textbf{Sfatul lui Titu Herdelea} declanșează punerea în aplicare a unui plan prin care Ion să intre în posesia pământurilor lui Vasile Baciu: \qu{Dacă nu vrea el să ți-o dea de bunăvoie, trebuie să-l silești}. Astfel, Ion o seduce pe Ana, vizitând-o câteva seri la rând, după plecarea lui George. Faptul că Ana rămâne însărcinată reprezintă un \textbf{stigmat} greu de suportat pentru o fată nemăritată.

Furia lui Vasile Baciu se dezlănțuie în momentul în care află de la George Bulbuc că Ion este responsabil pentru sarcina Anei. Urmează câteva secvențe de o violență greu de imaginat, în care Ana este bătută și alungată, când de tatăl ei, când de Ion. Acesta îi condiționează prezența de pământurile lui Vasile Baciu.

\textbf{Lăcomia lui Ion} este pusă în evidență de \textbf{negocierile} repetate și tensionate cu Vasile Baciu. Inițial, tatăl Anei este hotărât să nu-i cedeze lui Ion niciun loc, decât după ce va muri, propunere refuzată vehement de Ion. O altă negociere este cea în care în care Baciu îi oferă o parte din locurile de pământ, propunere urmată din nou de refuzul lui Ion. Preotul Belciug intervine ca mediator, demersul său fiind unul interesat, deoarece spera să obțină ceva din averea lui Vasile Baciu. De aceea, contractul încheiat va conține o clauză conform căreia, în lipsa urmașilor lui Ion, toate pământurile trec în proprietatea bisericii.

\textbf{La nuntă}, Ana are o revelație tragică, conștientizând faptul că Ion o iubește pe Florica. Sugestivă pentru starea Anei din acel moment este comparația \qu{tresări ca mușcată de viperă}. Lamentația repetată \qu{Norocul meu, norocul meu!} invocă un noroc inexistent.

Protagonistul romanului simte pentru prima dată mândria de stăpân în momentul în care devine proprietarul legitim al tuturor pământurilor lui Vasile Baciu. O \textbf{scenă simbolică} care ilustrează cel mai bine \textbf{patima} lui Ion \textbf{pentru pământ} este aceea în care protagonistul, îmbrăcat în straie de sărbătoare, se apleacă și sărută pământul: \qu{își lipi buzele cu voluptate de pământul ud}. În secvența aceasta, pământul nu este doar o simplă proprietate, ci o iubită pentru care protagonistul sacrifică tot.

\textbf{La nunta lui George Bulbuc cu Florica}, Ana simte din nou umilința femeii trădate. De data aceasta, \qu{rușinea crâncenă} se transformă într-o \qu{greață năbușitoare}. Sinuciderea Anei este anticipată de câteva semne și scene: imaginea apei tulburi ca metaforă a morții, moartea lui Avrum și a lui Dumitru Moarcăș.

Nici moartea Anei, nici pierderea lui Petrișor nu i-au provocat lui Ion grave mustrări de conștiință. Singura suferință este produsă de pierderea pământurilor.

Odată ce glasul pământului a încetat, se aude din nou glasul iubirii: Ion își redirecționează instinctul de posesiune asupra Floricăi, folosind prietenia cu George drept pretext pentru a fi cât mai mult în preajma ei. Savista, oloaga satului, este cea care îl avertizează pe George. Savista este un \textbf{personaj marcat de destin}, care grăbește finalul protagonistului ca măsură justițiară pentru suferința provocată Anei.
\section{Concluzie}
În concluzie, romanul lui Rebreanu este un amestec de vechi și nou, de tradiționalism și de accente antisămănătoriste care prezintă într-o manieră originală încercările unui țăran sărac de a-și depăși statutul social.

\part*{Caracterizare}
\setcounter{section}{0}
\section{Introducere}
Protagonistul romanului \qu{Ion} de Liviu Rebreanu rămâne cunoscut în literatura română drept \textbf{simbolul țăranului împătimit de pământ}.

\textbf{Fiind un personaj complex}, el a stârnit reacții diverse din partea criticilor literari. De exemplu, pentru George Călinescu, Ion \qu{nu e decât o brută căreia șiretenia îi ține loc de deșteptăciune}. Nicolae Manolescu reinterpretează afirmația lui Călinescu, adăugând că Ion este \qu{o brută ingenuă}. Cu alte cuvinte, Ion se comportă cum au făcut-o și alți țărani din Pripas.

\textbf{Ca personaj realist}, Ion se încadrează în tipologia țăranului, individualizându-se prin mijloacele prin care intră în posesia pământului.

\section{Cuprinsul}
Protagonistul romanului este caracterizat atât direct, cât și indirect, prin fapte, comportament și relația cu celelalte personaje.

Profilul personajului se conturează încă de la începutul romanului, când \textbf{naratorul}, cu o \textbf{fină intuiție psihologică}, descrie oscilarea \textbf{între cele două glasuri}: glasul iubirii simbolizat de Florica, \qu{mai frumoasă ca oricând, dar săracă}; și glasul pământului simbolizat de Ana, \qu{urâțică, dar cu locuri și case și vite multe}.

Conflictul de la horă cu Vasile Baciu și conflictul de la cârciumă cu George Bulbuc atestă \textbf{caracterul umil al țăranului sărac}. Jignirile lui Vasile Baciu îl dor cumplit pe Ion: \qu{tâlharule}, \qu{sărăntocule}, \qu{hoțule}.

Ion este un \textbf{personaj complex}, alcătuit din lumini și umbre, adică din calități și defecte. El trezește simpatia familiei Herdelea prin \textbf{muncă} și \textbf{istețime}. De exemplu, \textbf{doamna Herdelea îl caracterizează direct} în felul următor: \qu{Ion e băiat cumsecade. E muncitor, harnic, isteț}.

Deși sărac, Ion este \qu{\textbf{iute și harnic}, ca mă-sa}, \textbf{așa cum remarcă} și \textbf{naratorul}, iubind munca și pământul. Din dragoste pentru pământ, Ion renunță la școală.

Pentru \textbf{comportamentul agresiv} de la cârciumă, Ion este certat de preotul Belciug, care îl caracterizează direct astfel: \qu{Ești un stricat și un bătăuș și un om de nimic!}. Ion se simte umilit și vrea să se răzbune, de aceea îl roagă pe învățătorul Ion Herdelea să redacteze o plângere adresată Ministerului Justiției din care să reiasă că Ion este victima preotului și a altor țărani din sat.

\textbf{Cu luciditate}, Ion concepe un \textbf{plan} prin care să ajungă în posesia pământurilor lui Vasile Baciu. Ion împrumută mentalitatea și comportamentul comunității din care face parte. \textbf{Alianța cu o fată cu zestre} este o strategie frecvent întâlnită în satul Pripas. Așa a procedat și Vasile Baciu, și Alexandru Glanetașu.

Conform planului, Ion se folosește de Ana, pe care o seduce și o lasă însărcinată. Din acest moment, \textbf{faptele} protagonistului devin \textbf{etape ale dezumanizării}.

\textbf{Ion este caracterizat indirect prin relație cu Ana}. Inițial, Ion este \textbf{viclean}, făcând-o pe Ana să se îndrăgostească, pentru ca, pe urmă, să se distanțeze de ea. În episoadele în care o bate și o alungă de acasă este \textbf{agresiv, crud, nemilos}. Cu \textbf{cinism} îi condiționează prezența de pământurile lui Vasile Baciu.

\textbf{Lăcomia protagonistului} reiese din episodul în care intră cu plugul pe bucata de pământ a lui Simion Lungu, pretextând că aparținuse odinioară familiei lui. Aceeași trăsătură reiese și din negocierile tensionate cu Vasile Baciu. \textbf{Avid fiind}, Ion este \textbf{sfidător} și \textbf{amenințător} la început, spunându-i lui Vasile Baciu că-l va da în judecată pentru că nu și-a respectat promisiunea. Prin urmare, Ion este \textbf{aspru} și \textbf{brutal} cu toți cei pe care îi consideră vinovați pentru condiția sa de țăran sărac.

Ion simte \textbf{mândria de stăpân} în momentul în care devine proprietarul legitim al pământurilor lui Vasile Baciu. Este sugestivă \textbf{remarca naratorului}: \qu{este atât de puternic încât are iluzia că poate să domnească peste tot cuprinsul}.

Protagonistul romanului este definit de \textbf{patima sa pentru pământ}. Din modul în care personajul se raportează la acest personaj simbolic, se conturează următoarele \textbf{ipostaze ale pământului}. \textsl{\textbf{Pământul ca mamă}} se conturează în copilărie, atunci când Ion renunță la școală ca să muncească pământul, așa cum precizează și naratorul: \qu{pământul i-a fost mai drag decât o mamă}. Din acest motiv, criticul literar George Călinescu afirmă despre Ion că este o \qu{ființă telurică}. \textsl{\textbf{Pământul ca zeitate atotputernică}} reiese din secvența în care Ion se simte mic în fața imensității uriașului. \textsl{\textbf{Pământul ca ființă iubită}} reiese din scena simbolică în care Ion, îmbrăcat în straie de sărbătoare se apleacă și \qu{își lipi buzele cu voluptate de pământul ud}.

Ion \textbf{nu are mustrări de conștiință} nici după moartea Anei, nici după moartea lui Petrișor. Singura suferință este provocată de pierderea pământurilor, care conform contractului încheiat, trec în proprietatea bisericii.

\textbf{Cu abilitate}, Ion își redirecționează instinctul de posesiune de la pământ către Florica. În acest sens, prietenia cu George este folosită ca pretext pentru a fi cât mai mult în preajma Floricăi.

\textbf{Finalul protagonistului este motivat moral}. Savista este cea care îi grăbește finalul, ca măsură justițiară pentru suferința provocată Anei.
\section{Concluzie}
În concluzie, romanul pare a sugera că vina, dacă există una, trebuie căutată în Ion, despre care Nicolae Manolescu afirmă că este \qu{un potențial factor de dezordine socială}.
\end{document}
